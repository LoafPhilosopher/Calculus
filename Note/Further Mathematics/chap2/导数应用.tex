\ifx\allfiles\undefined
\documentclass[12pt, a4paper, oneside, UTF8]{ctexbook}  % +  这一句是新增加的
\input{../config/config}
\begin{document}
\begin{sloppypar}

    % \title{{\Huge{\textbf{考研高等数学笔记}}}}
\author{作者:LoafPhilosopher }
\date{\today}
\maketitle                   % 在单独的标题页上生成一个标题
\thispagestyle{empty}        % 前言页面不使用页码
% \begin{center}
% 	\Huge\textbf{前言}
% \end{center}
% 本笔记主要结合张宇老师和武忠祥老师的以及邂逅遗憾老师的内容进行编写
% \begin{flushright}
% 	\begin{tabular}{c}
% 		\today \newline 
% 	\end{tabular}
% \end{flushright}
\newpage                      % 新的一页
\pagestyle{plain}             % 设置页眉和页脚的排版方式(plain:页眉是空的,页脚只包含一个居中的页码)
\setcounter{page}{1}          % 重新定义页码从第一页开始
\pagenumbering{Roman}         % 使用大写的罗马数字作为页码

\begin{spacing}{1.5}
	\tableofcontents
\end{spacing}           % 生成目录
\newpage                      % 以下是正文
\pagestyle{plain}
\setcounter{page}{1}          % 使用阿拉伯数字作为页码
\pagenumbering{arabic}
% \setcounter{chapter}{-1}    % 设置 -1 可作为第零章绪论从第零章开始 
    \else
    \fi
    %  ############################ 正文部分
    \chapter{导数的应用}

    \section{极值与最值的概念}

    \subsection{极值}
    \begin{defn}{广义的极值}{}

    \end{defn}

    \begin{defn}{真正的极值}{}

    \end{defn}

    \subsection{最值}
    \begin{defn}{广义的最值}{}

    \end{defn}

    \begin{defn}{真正的最值}{}

    \end{defn}

    \section{单调性与极值的判别}

    \subsection{单调性的判断}

    \subsection{极值的判别}

    \section{凹凸性与拐点}

    \subsection{凹凸性与拐点的定义}
    \begin{defn}{凹凸性的定义}{}

    \end{defn}
    \begin{defn}{拐点的定义}{}

    \end{defn}
    \subsection{凹凸性与拐点的判别}

    \section{渐进线}

    \subsection{铅垂渐进线}
    \begin{defn}{铅锤渐近线}{}

    \end{defn}
    \subsection{水平渐近线}
    \begin{defn}{水平渐近线}{}

    \end{defn}
    \subsection{斜渐进线}
    \begin{defn}{斜渐进线}{}

    \end{defn}
    %  ############################ 正文部分
    \ifx\allfiles\undefined
\end{sloppypar}
\end{document}
\fi
