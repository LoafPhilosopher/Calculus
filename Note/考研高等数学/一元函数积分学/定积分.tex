\ifx\allfiles\undefined
\documentclass[8pt a4paper, oneside, UTF8]{ctexbook}  % +  这一句是新增加的
\input{../config/config}
\begin{document}
\begin{sloppypar}
    % \title{{\Huge{\textbf{考研高等数学笔记}}}}
\author{作者:LoafPhilosopher }
\date{\today}
\maketitle                   % 在单独的标题页上生成一个标题
\thispagestyle{empty}        % 前言页面不使用页码
% \begin{center}
% 	\Huge\textbf{前言}
% \end{center}
% 本笔记主要结合张宇老师和武忠祥老师的以及邂逅遗憾老师的内容进行编写
% \begin{flushright}
% 	\begin{tabular}{c}
% 		\today \newline 
% 	\end{tabular}
% \end{flushright}
\newpage                      % 新的一页
\pagestyle{plain}             % 设置页眉和页脚的排版方式(plain:页眉是空的,页脚只包含一个居中的页码)
\setcounter{page}{1}          % 重新定义页码从第一页开始
\pagenumbering{Roman}         % 使用大写的罗马数字作为页码

\begin{spacing}{1.5}
	\tableofcontents
\end{spacing}           % 生成目录
\newpage                      % 以下是正文
\pagestyle{plain}
\setcounter{page}{1}          % 使用阿拉伯数字作为页码
\pagenumbering{arabic}
% \setcounter{chapter}{-1}    % 设置 -1 可作为第零章绪论从第零章开始 
    \else
    \fi
    %  ############################ 正文部分
    \chapter{定积分(黎曼积分)}
    \section{定积分的概念}
    \begin{defn}{定积分的定义}{}
        若函数$f(x)$在区间$[a,b]$上有界,在$(a,b)$上任取$n-1$ 个分点$x_i(i=1,2,3,\cdots,n-1)$,定义$x_0=a$ 和$x_n=b$,且$a=x_0<x_1<x_2<x_3<\cdots<x_{n-1}<x_n=b$,记$\Delta x_k=x_k-x_{k-1},k=1,2,3,\cdots,n$.并任取一点$\xi_{k}\in[x_{k-1},x_{k}]$,记$\lambda=\max_{1\leq k \leq n}\{\Delta x_{k}\}$,若当$\lambda\to0$时,极限$\lim_{\lambda\to0}\sum_{k=1}^{n}f(\xi_{k})\Delta x_{k}$存在且与分点$x_i$及点$\xi_k$的取法无关,则称函数$f(x)$在区间$[a,b]$上可积,即
        $$
            \int_{a}^{b}f(x)\mathrm{d}x=\lim_{\lambda\to0}\sum_{k=1}^{n}f\left(\xi_{k}\right)\Delta x_{k}
        $$
    \end{defn}
    \begin{criterion}{定积分的几何意义}{}
        在$[a,b]$上,
        \begin{enumerate}
            \item 若$f(x)\geqslant0$,定积分$\int_a^bf(x) d x$表示由曲线$y=f(x)$、直线$x=a$、直线$x=b$与$x$轴所围成的曲边梯形的面积
            \item 若$f(x)\leqslant0$,定积分$\int_a^bf(x) d x$ 表示由曲线$y=f(x)$、直线$x=a$、直线 $x=b$ 与 $x$ 轴所围成的曲边梯形面积的负值
            \item 若$f(x)$既有正值又有负值(如下图所示),定积分$\int_a^bf(x) dx$表示$x$轴上方图形的面积减去$x$轴下方图形的面积.
        \end{enumerate}
    \end{criterion}
    \begin{defn}{定积分的精确定义}{}
        当定积分存在时,存在两个“任取”:分点$x_i$任取,一点$\xi_i\in(x_{i-1},x_i)$任取.故可作两个“特取”:将$[a,b]n$等分且取每个小区间的右端点为$\xi_i$,即
        $$
            \int_{a}^{b}f(x)\mathrm{d}x=\lim_{n\to\infty}\sum_{i=1}^{n}f\Bigg(a+\dfrac{b-a}{n}i\Bigg)\frac{b-a}{n}
        $$
        \begin{figure}[H]
            \centering
            \includegraphics[width=0.4\textwidth]{7.1.1.pdf}
        \end{figure}
        若将式子中的$a,b$ 特殊化为 0,1 这两个数,得出的形式更为简单:
        $$
            \int_0^1f(x)\mathrm{d}x=\lim_{n\to\infty}\sum_{i=1}^nf\biggl(\frac{i}{n}\biggr)\frac{1}{n}\:.
        $$
    \end{defn}
    \begin{them}{定积分的存在定理}{}
        定积分的存在性,也称一元函数的(常义)可积性.这里的“常义”是指“区间有限,函数有界”, 也有人称为黎曼可积性\footnote{此处的可积性与后面的反常积分不同,反常积分是“区间无穷,函数无穷”}.
        \begin{itemize}
            \item 定积分存在的充分条件
                  \begin{enumerate}
                      \item 若$f(x)$在$[a,b]$上连续,则$\int_a^bf(x) dx$存在
                      \item 若$f(x)$在$[a,b]$上单调,则$\int_a^bf(x) dx$存在
                      \item 若$f(x)$在$[a,b]$上\textbf{有界,且只有有限个间断点}\footnote{其实只剩下了可去和跳跃间断点},则$\int_a^bf(x)dx$存在
                            \newline 对于上三个充分条件,可以结合定积分的几何意义来理解,定积分存在就是指的围成的曲边梯形面积能算出来.只要围出来的面积可以算出来,即使$f(x)$在有限个点的函数值发生突变,$f(x)$依然可积.因为在间断点,那条线与$x$轴围成的面积为0,所以第一类间断点不影响总体面积.
                            \begin{center}
                                \tikzset{every picture/.style={line width=0.75pt}} %set default line width to 0.75pt    
                                \begin{tikzpicture}[x=0.75pt,y=0.75pt,yscale=-1,xscale=1]
                                    \draw  (14.71,111.41) -- (207.36,111.41)(49.75,23) -- (49.75,132.3) (200.36,106.41) -- (207.36,111.41) -- (200.36,116.41) (44.75,30) -- (49.75,23) -- (54.75,30) (69.75,106.41) -- (69.75,116.41)(89.75,106.41) -- (89.75,116.41)(109.75,106.41) -- (109.75,116.41)(129.75,106.41) -- (129.75,116.41)(149.75,106.41) -- (149.75,116.41)(169.75,106.41) -- (169.75,116.41)(189.75,106.41) -- (189.75,116.41)(29.75,106.41) -- (29.75,116.41)(44.75,91.41) -- (54.75,91.41)(44.75,71.41) -- (54.75,71.41)(44.75,51.41) -- (54.75,51.41) ;
                                    \draw   ;
                                    \draw [line width=1.5]    (63.36,87) .. controls (92.36,41) and (148.36,91) .. (170.36,34) ;
                                    \draw  [dash pattern={on 4.5pt off 4.5pt}]  (63.36,87) -- (63.36,117) ;
                                    \draw  [dash pattern={on 4.5pt off 4.5pt}]  (170.36,34) -- (170.36,117) ;
                                    \draw  (11.71,257.41) -- (217.36,257.41)(49.12,169) -- (49.12,278.3) (210.36,252.41) -- (217.36,257.41) -- (210.36,262.41) (44.12,176) -- (49.12,169) -- (54.12,176) (69.12,252.41) -- (69.12,262.41)(89.12,252.41) -- (89.12,262.41)(109.12,252.41) -- (109.12,262.41)(129.12,252.41) -- (129.12,262.41)(149.12,252.41) -- (149.12,262.41)(169.12,252.41) -- (169.12,262.41)(189.12,252.41) -- (189.12,262.41)(29.12,252.41) -- (29.12,262.41)(44.12,237.41) -- (54.12,237.41)(44.12,217.41) -- (54.12,217.41)(44.12,197.41) -- (54.12,197.41) ;
                                    \draw   ;
                                    \draw [line width=1.5]    (60.36,233) .. controls (74.18,212.17) and (100.87,234.81) .. (111.36,209) ;
                                    \draw  [dash pattern={on 4.5pt off 4.5pt}]  (60.36,233) -- (60.36,263) ;
                                    \draw  [dash pattern={on 4.5pt off 4.5pt}]  (196.36,172) -- (196.36,255) ;
                                    \draw [line width=1.5]    (112.36,202) .. controls (183.36,201) and (154.36,171) .. (196.36,172) ;
                                    \draw  [dash pattern={on 4.5pt off 4.5pt}]  (112.36,202) -- (112.36,259) ;
                                    \draw   (83.71,222.82) .. controls (83.71,221.44) and (84.69,220.32) .. (85.89,220.32) .. controls (87.1,220.32) and (88.07,221.44) .. (88.07,222.82) .. controls (88.07,224.2) and (87.1,225.32) .. (85.89,225.32) .. controls (84.69,225.32) and (83.71,224.2) .. (83.71,222.82) -- cycle ;
                                    \draw  [fill={rgb, 255:red, 0; green, 0; blue, 0 }  ,fill opacity=1 ] (108.21,169.82) .. controls (108.21,168.44) and (109.19,167.32) .. (110.39,167.32) .. controls (111.6,167.32) and (112.57,168.44) .. (112.57,169.82) .. controls (112.57,171.2) and (111.6,172.32) .. (110.39,172.32) .. controls (109.19,172.32) and (108.21,171.2) .. (108.21,169.82) -- cycle ;
                                    \draw   (150.46,195.82) .. controls (150.46,194.44) and (151.44,193.32) .. (152.64,193.32) .. controls (153.85,193.32) and (154.82,194.44) .. (154.82,195.82) .. controls (154.82,197.2) and (153.85,198.32) .. (152.64,198.32) .. controls (151.44,198.32) and (150.46,197.2) .. (150.46,195.82) -- cycle ;
                                    \draw    (205.33,57.33) -- (296.7,108.36) ;
                                    \draw [shift={(298.44,109.33)}, rotate = 209.18] [color={rgb, 255:red, 0; green, 0; blue, 0 }  ][line width=0.75]    (10.93,-3.29) .. controls (6.95,-1.4) and (3.31,-0.3) .. (0,0) .. controls (3.31,0.3) and (6.95,1.4) .. (10.93,3.29)   ;
                                    \draw    (205.11,185.33) -- (298.56,151.35) ;
                                    \draw [shift={(300.44,150.67)}, rotate = 160.02] [color={rgb, 255:red, 0; green, 0; blue, 0 }  ][line width=0.75]    (10.93,-3.29) .. controls (6.95,-1.4) and (3.31,-0.3) .. (0,0) .. controls (3.31,0.3) and (6.95,1.4) .. (10.93,3.29)   ;
                                    \draw (661,128) node [anchor=north west][inner sep=0.75pt]   [align=left] {$\displaystyle y$};
                                    \draw (212,102) node [anchor=north west][inner sep=0.75pt]   [align=left] {$\displaystyle x$};
                                    \draw (32.4,112.92) node [anchor=north west][inner sep=0.75pt]   [align=left] {$\displaystyle O$};
                                    \draw (-46,78) node [anchor=north west][inner sep=0.75pt]   [align=left] {$ $};
                                    \draw (30.4,17) node [anchor=north west][inner sep=0.75pt]   [align=left] {$\displaystyle y$};
                                    \draw (59,118) node [anchor=north west][inner sep=0.75pt]   [align=left] {$\displaystyle a$};
                                    \draw (165,120) node [anchor=north west][inner sep=0.75pt]   [align=left] {$\displaystyle b$};
                                    \draw (222,249) node [anchor=north west][inner sep=0.75pt]   [align=left] {$\displaystyle x$};
                                    \draw (29.4,258.92) node [anchor=north west][inner sep=0.75pt]   [align=left] {$\displaystyle O$};
                                    \draw (27.4,163) node [anchor=north west][inner sep=0.75pt]   [align=left] {$\displaystyle y$};
                                    \draw (56,264) node [anchor=north west][inner sep=0.75pt]   [align=left] {$\displaystyle a$};
                                    \draw (191,264) node [anchor=north west][inner sep=0.75pt]   [align=left] {$\displaystyle b$};
                                    \draw (311.33,110.67) node [anchor=north west][inner sep=0.75pt]   [align=left] {$\displaystyle \int _{a}^{b} f( x) dx$};
                                \end{tikzpicture}
                            \end{center}
                  \end{enumerate}
            \item 定积分存在的必要条件
                  \begin{enumerate}
                      \item 可积函数必有界,即若定积分$\int_a^bf(x)dx$存在,则$f(x)$在$[a,b]$上必有界. \footnote{当我们任意分割图形底边为若干小段时若$f(x)$在区间$[a、b]$上无界,则至少存在一个小段$\Delta x$,使得“面积”$f(x) \Delta x$无穷大,这样整个曲边梯形的面积就是无穷大,则极限就不存在,所以可积函数必有界}
                  \end{enumerate}
        \end{itemize}
    \end{them}
    \begin{criterion}{定积分的性质}{}
        当$b=a$时,$\int_a^af(x)dx=0$;
        当$a>b$时,$\int_a^bf(x)$d$x=-\int_b^af(x)dx$
        \begin{enumerate}
            \item 求区间长度:假设 $a<b$,则$\int _a^b dx=b-a=L$ ,其中$L$ 为区间$[a,b]$的长度
            \item 积分的线性性质:设$k_{_1}, k_{_2}$为常数,则$\int_{a}^{b}[k_{_1}f(x)\pm k_{_2}g(x)]\mathrm{d}x=k_{_1}\int_{a}^{b}f(x)\mathrm{d}x\pm k_{_2}\int_{a}^{b}g(x)\mathrm{d}x$.
            \item 积分的可加(拆)性:无论$a,b,c$的大小如何,总有$\int_a^bf(x)\mathrm{d}x=\int_a^cf(x)\mathrm{d}x+\int_c^bf(x)\mathrm{d}x$
            \item 积分的保号性:若在区间$[a,b]$上$f(x)\leqslant g(x)$,则有$\int_{a}^{b}f(x)\mathrm{d}x\leqslant\int_{a}^{b}g(x)\mathrm{d}x$.特殊地,有$$\quad\left|\int_a^bf(x)\mathrm{d}x\right|\leqslant\int_a^b\left|f(x)\right|\mathrm{d}x$$
            \item 估值定理:设$M,m$分别是$f(x)$在$[a,b]$上的最大值和最小值,$L$为区间$[a,b]$的长度,则有
                  $$mL\leqslant\int_{a}^{b}f(x)\mathrm{d}x\leqslant ML\:$$
        \end{enumerate}
    \end{criterion}
    \section{定积分的计算}
    \begin{defn}{牛顿莱布尼茨公式}{}
        设函数 $F(x)$是连续函数$f(x)$在$[a,b]$上的一个原函数,则
        $$\int_{a}^{b}f(x)\mathrm{d}x=F(x)\Big|_{a}^{b}=F(b)-F(a)$$
    \end{defn}
    \begin{criterion}{牛顿莱布尼茨公式推广}{}
        \begin{itemize}
            \item 若$f(x)$在$\left[a,b\right]$上有原函数$F(x)$,则$\int_{a}^{b}f(x)\mathrm{d}x=F(b)-F(a)$\footnote{有原函数即可,不需要连续}
            \item 若$f(x)$在$[a,b]$上分段有原函数,如$[a,c)$上有原函数$F_1(x),(c,b]$上有原函数$F_2(x)$,则
                  $$\int_{a}^{b}f(x)\mathrm{d}x=\int_{a}^{c}f(x)\mathrm{d}x+\int_{c}^{b}f(x)\mathrm{d}x=F_{1}(c-0)-F_{1}(a)+F_{2}(b)-F_{2}(c+0)\:.$$\footnote{其实就是以$C$为中心拆成两半进行计算}
                  若$F_1(c-0),F_2(c+0)$存在,则$\int_a^bf(x)dx$收敛。若$F_1(c-0),F_{2}(c+0)$至少有一个不存在,则$\int_a^bf(x)$d$x$发散。
        \end{itemize}
    \end{criterion}
    \section{定积分积分法}
    \subsection{定积分的换元积分法}
    设$f(x)$在$[a,b]$上连续,函数$x=\varphi(t)$满足
    \begin{enumerate}
        \item $\varphi(\alpha)=a,\varphi(\beta)=b$
        \item $x=\varphi(t)$在$[\alpha,\beta]$或$[\beta,\alpha]$上有连续的导数,且其值域为$R_\varphi=[a,b]$,则有$$\int_{a}^{b}f(x)\mathrm{d}x=\int_{\alpha}^{\beta}f[\varphi(t)]\varphi'(t)\mathrm{d}t\:.$$
    \end{enumerate}
    \subsection{定积分的分部积分法}
    $\int_{a}^{b}u(x)v^{\prime}(x)$d$x=u(x)v(x)\Big|_a^{b}-\int_{a}^{b}\nu(x)u^{\prime}(x)$d$x$,这里要求$u^\prime(x),v^{\prime}(x)$在$[a,b]$上连续。
    \begin{conclusion}{定积分分部积分法常用结论}{}
        \begin{itemize}
            \item 区间再现公式:设$f(x)$为连续函数,则$$\int_{a}^{b}f(x)dx=\int_{a}^{b}f(a+b-x)dx$$
            \item $\int_{0}^{\frac{\pi}{2}}\sin^{n}x\mathrm{d}x=\int_{0}^{\frac{\pi}{2}}\cos^{n}x\mathrm{d}x=\begin{cases}\dfrac{n-1}{n} \cdot \dfrac{n-3}{n-2} \cdot \cdots \cdot \dfrac{2}{3} \cdot 1,&n\text{ 为大于 1 的奇数,}\\ \dfrac{n-1}{n} \cdot \dfrac{n-3}{n-2} \cdot \cdots \cdot \dfrac{1}{2} \cdot \dfrac{\pi}{2},&n\text{ 为正偶数 }.\end{cases}$
            \item $\begin{aligned} & \int_0^\pi \sin ^n x \mathrm{~d} x= \begin{cases}2 \cdot \dfrac{n-1}{n} \cdot \dfrac{n-3}{n-2} \cdots \cdot \dfrac{2}{3} \cdot 1, & n \text {为大于} 1 \text {的奇数, } \\ 2 \cdot \dfrac{n-1}{n} \cdot \dfrac{n-3}{n-2} \cdots \cdots \cdot \dfrac{1}{2} \cdot \dfrac{\pi}{2}, & n \text { 为正偶数, }\end{cases} \\ & \int_0^\pi \cos ^n x \mathrm{~d} x= \begin{cases}0, & n \text { 为正奇数, } \\ 2 \cdot \dfrac{n-1}{n} \cdot \dfrac{n-3}{n-2} \cdots \cdots \dfrac{1}{2} \cdot \dfrac{\pi}{2}, & n \text { 为正偶数. }\end{cases} \end{aligned}$
            \item $\int_{0}^{2\pi}\cos^{n}x\mathrm{d}x=\int_{0}^{2\pi}\sin^{n}x\mathrm{d}x=\begin{cases}0\:,&n\text{ 为正奇数,}\\4\cdot\dfrac{n-1}{n}\cdot\dfrac{n-3}{n-2}\cdot\cdots\cdot\dfrac{1}{2}\cdot\dfrac{\pi}{2}\:,&n\text{ 为正偶数 }.\end{cases}$
        \end{itemize}
    \end{conclusion}
    %  ############################ 正文部分
    \ifx\allfiles\undefined
\end{sloppypar}
\end{document}
\fi
