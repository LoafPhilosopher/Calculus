\ifx\allfiles\undefined
\documentclass[8pt a4paper, oneside, UTF8]{ctexbook} 
\input{../config/config}
\begin{document}
\begin{sloppypar}
    % \title{{\Huge{\textbf{考研高等数学笔记}}}}
\author{作者:LoafPhilosopher }
\date{\today}
\maketitle                   % 在单独的标题页上生成一个标题
\thispagestyle{empty}        % 前言页面不使用页码
% \begin{center}
% 	\Huge\textbf{前言}
% \end{center}
% 本笔记主要结合张宇老师和武忠祥老师的以及邂逅遗憾老师的内容进行编写
% \begin{flushright}
% 	\begin{tabular}{c}
% 		\today \newline 
% 	\end{tabular}
% \end{flushright}
\newpage                      % 新的一页
\pagestyle{plain}             % 设置页眉和页脚的排版方式(plain:页眉是空的,页脚只包含一个居中的页码)
\setcounter{page}{1}          % 重新定义页码从第一页开始
\pagenumbering{Roman}         % 使用大写的罗马数字作为页码

\begin{spacing}{1.5}
	\tableofcontents
\end{spacing}           % 生成目录
\newpage                      % 以下是正文
\pagestyle{plain}
\setcounter{page}{1}          % 使用阿拉伯数字作为页码
\pagenumbering{arabic}
% \setcounter{chapter}{-1}    % 设置 -1 可作为第零章绪论从第零章开始 
    \else
    \fi
    %  ############################ 正文部分
    \chapter{不定积分}

    \section{不定积分的概念}
    \begin{defn}{原函数与不定积分的定义}{}
        设函数$f(x)$定义在某区间$I$上,若存在可导函数$F(x)$,对于该区间上任意一点都有$F^\prime(x)=f(x)$成立,则称 $F(x)$是$f(x)$在区间 $I$ 上的一个原函数.称$\int f(x)dx=F(x)+C$ 为$f(x)$在区间$I$上的不定积分.
    \end{defn}
    \begin{them}{原函数(不定积分)存在定理}{}
        \begin{enumerate}
            \item 连续函数$f(x)$必有原函数$F(x)$.\\
                  已知$f(x)=F'(x)$,那么如果导数连续,原函数一定存在.
        \end{enumerate}
    \end{them}

    \section{不定积分的计算}

    %  ############################ 正文部分
    \ifx\allfiles\undefined
\end{sloppypar}
\end{document}
\fi
