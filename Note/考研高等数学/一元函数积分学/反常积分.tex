\ifx\allfiles\undefined
\documentclass[8pt a4paper, oneside, UTF8]{ctexbook}  % +  这一句是新增加的
\usepackage{amsmath}   % 数学公式
\usepackage[dvipsnames]{xcolor}
\usepackage{amsthm}    % 定理环境
\usepackage{amssymb}   % 更多公式符号
\usepackage{graphicx}  % 插图
\usepackage{mathrsfs}  % 数学字体
\usepackage{enumitem}  % 列表
\usepackage{geometry}  % 页面调整
\usepackage{unicode-math}
\usepackage{extarrows}
\usepackage{subfigure}
\usepackage{extarrows}
\usepackage{footnote}
\usepackage{svg}
\usepackage[colorlinks,linkcolor=black]{hyperref}
\usepackage{supertabular}
\usepackage{tcolorbox}
\usepackage{ulem}
\usepackage{framed}
\usepackage{float}
\usepackage{microtype}
\newcommand{\arccot}{\mathrm{arccot}\,}
\tcbuselibrary{breakable}
\tcbuselibrary{most}
\newcounter{problemname}
\newenvironment{solution}{\par\noindent\textbf{解答. }}{\par}
\newenvironment{note}{\par\noindent\textbf{题目\arabic{problemname}的注记. }}{\par}
\definecolor{shadecolor}{RGB}{241, 241, 255}
\newenvironment{problem}{\begin{shaded}\stepcounter{problemname}\par\noindent\textbf{题目\arabic{problemname}. }}{\end{shaded}\par}

\graphicspath{ {figure/},{../figure/}, {config/}, {../config/} }  % 配置图形文件检索目录
\linespread{1.2} % 行高

% 页码设置
\geometry{top=25.4mm,bottom=25.4mm,left=20mm,right=20mm,headheight=2.17cm,headsep=4mm,footskip=12mm}

% 设置列表环境的上下间距
\setenumerate[1]{itemsep=5pt,partopsep=0pt,parsep=\parskip,topsep=5pt}
\setitemize[1]{itemsep=5pt,partopsep=0pt,parsep=\parskip,topsep=5pt}
\setdescription{itemsep=5pt,partopsep=0pt,parsep=\parskip,topsep=5pt}

% 定理环境
% ########## 定理环境 start ####################################

% #### 将 config.tex 中的定理环境的对应部分替换为如下内容
% 定义单独编号,其他四个共用一个编号计数 这里只列举了五种,其他可类似定义(未定义的使用原来的也可)
\newtcbtheorem[auto counter, number within=section, list type=subsubsection, list inside=toc]{defn}{定义}
{
    colback=green!5,colframe=green!35!black,fonttitle=\bfseries, title={Comment \thetcbcounter}, list entry={Comment \thetcbcounter\quad}, %标题
    breakable, %支持跨页
    before upper={\parindent10pt\noindent},  % 支持缩进。\noindent:首行不缩进
    % left = 2mm, %文字离线框左边的边距
    % right = 1mm,%同上
    % top = 1mm,%同上
    % bottom = 1mm,%同上
    % arc is angular = 1mm, % 棱角线框
    % sharp corners, % 直角线框
    % enhanced,frame hidden, % 隐藏线框
    % enhanced, drop fuzzy shadow,  % 显示阴影
}
{def}

\newtcbtheorem[auto counter, number within=section, list type=subsubsection, list inside=toc]{lemma}{引理}
{
    colback=SeaGreen!10!CornflowerBlue!10,colframe=RoyalPurple!55!Aquamarine!100!,fonttitle=\bfseries, title={Comment \thetcbcounter}, list entry={Comment \thetcbcounter\quad}, %标题
    breakable, %支持跨页
    before upper={\parindent10pt\noindent},  % 支持缩进。\noindent:首行不缩进
    % left = 2mm, %文字离线框左边的边距
    % right = 1mm,%同上
    % top = 1mm,%同上
    % bottom = 1mm,%同上
    % arc is angular = 1mm, % 棱角线框
    % sharp corners, % 直角线框
    % enhanced,frame hidden, % 隐藏线框
    % enhanced, drop fuzzy shadow,  % 显示阴影
}
{lem}


\newtcbtheorem[auto counter, number within=section, list type=subsubsection, list inside=toc]{them}{定理}
{
    colback=Salmon!20, colframe=Salmon!90!Black,fonttitle=\bfseries, title={Comment \thetcbcounter}, list entry={Comment \thetcbcounter\quad}, %标题
    breakable, %支持跨页
    before upper={\parindent10pt\noindent},  % 支持缩进。\noindent:首行不缩进
    % left = 2mm, %文字离线框左边的边距
    % right = 1mm,%同上
    % top = 1mm,%同上
    % bottom = 1mm,%同上
    % arc is angular = 1mm, % 棱角线框
    % sharp corners, % 直角线框
    % enhanced,frame hidden, % 隐藏线框
    % enhanced, drop fuzzy shadow,  % 显示阴影
}
{them}
\newtcbtheorem[auto counter, number within=section, list type=subsubsection, list inside=toc]{criterion}{注}
{
    colback=CornflowerBlue!10,colframe=RoyalPurple!55!Aquamarine!100!,fonttitle=\bfseries, title={Comment \thetcbcounter}, list entry={Comment \thetcbcounter\quad}, %标题
    breakable, %支持跨页
    before upper={\parindent10pt\noindent},  % 支持缩进。\noindent:首行不缩进
    % left = 2mm, %文字离线框左边的边距
    % right = 1mm,%同上
    % top = 1mm,%同上
    % bottom = 1mm,%同上
    % arc is angular = 1mm, % 棱角线框
    % sharp corners, % 直角线框
    % enhanced,frame hidden, % 隐藏线框
    % enhanced, drop fuzzy shadow,  % 显示阴影
}
{cri}

\newtcbtheorem[auto counter, number within=section, list type=subsubsection, list inside=toc]{corollary}{推论}
{
    colback=Emerald!10,colframe=cyan!40!black,fonttitle=\bfseries, title={Comment \thetcbcounter}, list entry={Comment \thetcbcounter\quad}, %标题
    breakable, %支持跨页
    before upper={\parindent10pt\noindent},  % 支持缩进。\noindent:首行不缩进
    % left = 2mm, %文字离线框左边的边距
    % right = 1mm,%同上
    % top = 1mm,%同上
    % bottom = 1mm,%同上
    % arc is angular = 1mm, % 棱角线框
    % sharp corners, % 直角线框
    % enhanced,frame hidden, % 隐藏线框
    % enhanced, drop fuzzy shadow,  % 显示阴影
}
{cor}
% colback=red!5,colframe=red!75!black

% ######### 定理环境 end  #####################################

% ↓↓↓↓↓↓↓↓↓↓↓↓↓↓↓↓↓ 以下是自定义的命令  ↓↓↓↓↓↓↓↓↓↓↓↓↓↓↓↓

% 用于调整表格的高度  使用 \hline\xrowht{25pt}
\newcommand{\xrowht}[2][0]{\addstackgap[.5\dimexpr#2\relax]{\vphantom{#1}}}

% 表格环境内长内容换行  
\newcommand{\tabincell}[2]{\begin{tabular}{@{}#1@{}}#2\end{tabular}}

% 使用\linespread{1.5} 之后 cases 环境的行高也会改变,重新定义一个 ca 环境可以自动控制 cases 环境行高
\newenvironment{ca}[1][1]{\linespread{#1} \selectfont \begin{cases}}{\end{cases}}
% 和上面一样
\newenvironment{vx}[1][1]{\linespread{#1} \selectfont \begin{vmatrix}}{\end{vmatrix}}

\def\d{\textup{d}} % 直立体 d 用于微分符号 dx
\def\R{\mathbb{R}} % 实数域
\newcommand{\bs}[1]{\boldsymbol{#1}}    % 加粗,常用于向量
\newcommand{\ora}[1]{\overrightarrow{#1}} % 向量

% 数学 平行 符号
\newcommand{\pll}{\kern 0.5em/\kern -0.8em /\kern 0.5em}

% 用于空行\myspace{1} 表示空一行 填 2 表示空两行  
\newcommand{\myspace}[1]{\par\vspace{#1\baselineskip}}

\begin{document}
\begin{sloppypar}
    % \title{{\Huge{\textbf{高等数学笔记}}}}
\author{作者:于家崇}
\date{\today}
\maketitle                   % 在单独的标题页上生成一个标题

\thispagestyle{empty}        % 前言页面不使用页码
\begin{center}
	\Huge\textbf{前言}
\end{center}

If a job is worth doing,it's worth doing well
\begin{flushright}
	\begin{tabular}{c}
		\today \\ 如果一件事值得去做,那就值得去做好
	\end{tabular}
\end{flushright}

\newpage                      % 新的一页
\pagestyle{plain}             % 设置页眉和页脚的排版方式(plain:页眉是空的,页脚只包含一个居中的页码)
\setcounter{page}{1}          % 重新定义页码从第一页开始
\pagenumbering{Roman}         % 使用大写的罗马数字作为页码
\tableofcontents              % 生成目录

\newpage                      % 以下是正文
\pagestyle{plain}
\setcounter{page}{1}          % 使用阿拉伯数字作为页码
\pagenumbering{arabic}
% \setcounter{chapter}{-1}    % 设置 -1 可作为第零章绪论从第零章开始 
    \else
    \fi
    %  ############################ 正文部分
    \chapter{反常积分}
    \section{反常积分的概念}
    \begin{defn}{反常积分的定义}{}
        \begin{itemize}
            \item \textbf{无穷区间}上反常积分的概念与敛散性
                  \newline
                  设$F(x)$是$f(x)$在相应区间上的一个原函数:
                  \begin{enumerate}
                      \item $$
                                \int_{a}^{+\infty}f(x)\mathrm{d}x=\lim_{x\to+\infty}F(x)-F(a)
                            $$若上述极限存在,则称反常积分收敛,否则称发散
                      \item $$\int_{-\infty}^{b}f(x)\mathrm{d}x=F(b)-\lim_{x\to-\infty}F(x)$$若上述极限存在,则称反常积分收敛,否则称发散
                      \item $$\int_{-\infty}^{+\infty}f(x)\mathrm{d}x=\int_{-\infty}^{x_{0}}f(x)\mathrm{d}x+\int_{x_{0}}^{+\infty}f(x)\mathrm{d}x$$若右端\textbf{两个积分都收敛}\footnote{两个积分必须都收敛,不能使用不存在+不存在=存在},则称反常积分收敛,否则称发散
                  \end{enumerate}
            \item \textbf{无界函数}的反常积分的概念与敛散性
                  \begin{defn}{瑕点的定义}{}
                      使$f(x)$在 $x_0$的邻域内无界的点即为瑕点,例如:$\lim_{x\to 0}\dfrac{1}{x}=\infty ,x=0$为函数的瑕点.
                  \end{defn}
                  设$F(x)$是$f(x)$在相应区间上的一个原函数,$x_0$为$f(x)$的瑕点.
                  \begin{enumerate}
                      \item 若 $x=a$ 是唯一瑕点,则
                            $$\int_a^bf(x)\mathrm{d}x=F(b)-\lim_{x\to a^+}F(x)$$
                            若上述极限存在,则称反常积分收敛,否则称发散. \footnote{前面定积分章节,要求$\int_a ^b f(x) dx$存在的必要条件是$f(x)$有界,但是此处似乎无界,也可使积分存在,似乎矛盾,需要注意的是,两者的积分不一样,前面要求有界的是黎曼积分,而此次可以无界的是反常积分,二者不同.}
                      \item 若 $x=b$ 是唯一瑕点,则
                            $$\int_a^bf(x)\mathrm{d}x=\lim_{x\to b^-}F(x)-F(a)$$
                            若上述极限存在,则称反常积分收敛,否则称发散.
                      \item 若 $x=c\in(a,b)$是唯一瑕点,则
                            $$\int_a^bf(x)\mathrm{d}x=\int_a^cf(x)\mathrm{d}x+\int_c^bf(x)\mathrm{d}x\:,$$
                            若右端两个积分都收敛,则称反常积分收敛,否则称发散.
                  \end{enumerate}
        \end{itemize}
    \end{defn}
    \begin{criterion}{敛散性的判定}{}
        \begin{itemize}
            \item 无穷区间
                  \begin{enumerate}
                      \item 比较判别法\footnote{可通过不等式放缩来证明}:设函数$f(x),g(x)$在区间$[a, +\infty )$上连续,并且$0\leqslant f(x)\leqslant g(x)(a\leqslant x<+\infty)$,则
                            \begin{itemize}
                                \item 当$\int_a^{+\infty}g(x) dx$收敛时,$\int_a^{+\infty} f(x) dx$收敛
                                \item 当$\int_a^{+\infty}f(x) dx$发散时,$\int_a^{+\infty} g(x) dx$发散
                            \end{itemize}
                      \item 比较判别法的极限形式\footnote{通过等价无穷小求极限来判断}:设函数$f(x),g(x)$在区间$[a,+\infty)$上连续,且$f(x)\geqslant0,g(x)>0,\lim_{x\to+\infty}\dfrac{f(x)}{g(x)}=\lambda$(有限或 $\infty)$,则
                            \begin{itemize}
                                \item 当$\lambda\neq0$且$\lambda\neq\infty$时$,\int_a^{+\infty} f(x)$d$x$与$\int_a^{+\infty} g(x)$d$x$有相同的敛散性
                                \item 当$\lambda=0$时,若$\int_{a}^{+\infty}g(x) dx$收敛,则$\int_a^{+\infty}f(x) dx$也收敛
                                \item 当$\lambda=\infty$时,若$\int_a^{+\infty} g(x)$d$x$发散,则$\int_a^{+\infty}f(x)$d$x$也发散
                            \end{itemize}
                  \end{enumerate}
            \item 无界函数
                  \begin{enumerate}
                      \item 比较判别法:设$f(x),g(x)$在$(a,b]$上连续,瑕点同为$x=a$,并且$0\leqslant f(x)\leqslant g(x)(a<x\leqslant b)$,则
                            \begin{itemize}
                                \item 当$\int_{a}^{b}g(x) dx$收敛时,$\int_a^{b}f(x) dx$收敛
                                \item 当$\int_{a}^{b}f(x) dx$ 发散时,$\int_a^{b}g(x) dx$发散
                            \end{itemize}
                      \item 比较判别法的极限形式:设$f(x),g(x)$在$(a,b]$上连续,瑕点同为$x=a$,并且$f(x)\geqslant0,g(x)>0(a<x\leqslant b),\lim_{x\to a^{+}}\dfrac{f(x)}{g(x)}=\lambda$(有限或 $\infty)$,则
                            \begin{itemize}
                                \item 当$\lambda\neq0$且$\lambda\neq\infty$时,$\int_{a}^{b}f(x)\mathrm{d}x$和$\int_{a}^{b}g(x)\mathrm{d}x$有相同的敛散性
                                \item 当$\lambda=0$时,若$\int_{a}^{b}g(x)\mathrm{d}x$收敛,则$\int_{a}^{b}f(x)\mathrm{d}x$也收敛
                                \item 当$\lambda=\infty$时,若$\int_{a}^{b}g(x)\mathrm{d}x$发散,则$\int_{a}^{b}f(x)\mathrm{d}x$也发散
                            \end{itemize}
                  \end{enumerate}
        \end{itemize}
    \end{criterion}
    \begin{conclusion}{两个重要结论}{}
        \begin{enumerate}
            \item $\int _0 ^1 \dfrac{1}{x^p} dx \begin{cases}
                          \text{收敛},0<p<1 \\
                          \text{发散},p \geq 1
                      \end{cases}$
                  \newline 由于$x\to 0$时,存在$\sin x \sim \cdots \sim x $,因此有$\int _0 ^1 \dfrac{1}{\sin x ^p} dx \begin{cases}
                          \text{收敛},0<p<1 \\
                          \text{发散},p \geq 1
                      \end{cases}$,凡是与$x$趋于0的“速度”一样的函数$f(x)$均可如上讨论
            \item $\int_{1}^{+\infty}\dfrac{1}{x^{p}}\mathrm{d}x\begin{cases}{\text{收敛,}p>1}\\{\text{发散,}p\leq1} \end{cases}$.如当$x\to+\infty$且$a>0$时,$ax+b$亦趋于$+\infty$,与$x$趋于$+\infty$的“速度”一样,当$ax+b\geq k>0$时,$\int_1^{+\infty}\dfrac1{(ax+b)^p} dx$依然满足$\begin{cases}\text{收敛,}p>1,\\\text{发散,}p\leq1.\end{cases}$
        \end{enumerate}
    \end{conclusion}
    \section{反常积分的计算}
    \begin{conclusion}{}{}
        //  TODO
    \end{conclusion}
    %  ############################ 正文部分
    \ifx\allfiles\undefined
\end{sloppypar}
\end{document}
\fi
