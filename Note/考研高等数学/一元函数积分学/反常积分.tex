\ifx\allfiles\undefined
\documentclass[8pt a4paper, oneside, UTF8]{ctexbook}  % +  这一句是新增加的
\input{../config/config}
\begin{document}
\begin{sloppypar}
    % \title{{\Huge{\textbf{考研高等数学笔记}}}}
\author{作者:LoafPhilosopher }
\date{\today}
\maketitle                   % 在单独的标题页上生成一个标题
\thispagestyle{empty}        % 前言页面不使用页码
% \begin{center}
% 	\Huge\textbf{前言}
% \end{center}
% 本笔记主要结合张宇老师和武忠祥老师的以及邂逅遗憾老师的内容进行编写
% \begin{flushright}
% 	\begin{tabular}{c}
% 		\today \newline 
% 	\end{tabular}
% \end{flushright}
\newpage                      % 新的一页
\pagestyle{plain}             % 设置页眉和页脚的排版方式(plain:页眉是空的,页脚只包含一个居中的页码)
\setcounter{page}{1}          % 重新定义页码从第一页开始
\pagenumbering{Roman}         % 使用大写的罗马数字作为页码

\begin{spacing}{1.5}
	\tableofcontents
\end{spacing}           % 生成目录
\newpage                      % 以下是正文
\pagestyle{plain}
\setcounter{page}{1}          % 使用阿拉伯数字作为页码
\pagenumbering{arabic}
% \setcounter{chapter}{-1}    % 设置 -1 可作为第零章绪论从第零章开始 
    \else
    \fi
    %  ############################ 正文部分
    \chapter{反常积分}
    \section{反常积分的概念}
    \begin{defn}{反常积分的定义}{}
        \begin{itemize}
            \item \textbf{无穷区间}上反常积分的概念与敛散性
                  \newline
                  设$F(x)$是$f(x)$在相应区间上的一个原函数:
                  \begin{enumerate}
                      \item $$
                                \int_{a}^{+\infty}f(x)\mathrm{d}x=\lim_{x\to+\infty}F(x)-F(a)
                            $$若上述极限存在,则称反常积分收敛,否则称发散
                      \item $$\int_{-\infty}^{b}f(x)\mathrm{d}x=F(b)-\lim_{x\to-\infty}F(x)$$若上述极限存在,则称反常积分收敛,否则称发散
                      \item $$\int_{-\infty}^{+\infty}f(x)\mathrm{d}x=\int_{-\infty}^{x_{0}}f(x)\mathrm{d}x+\int_{x_{0}}^{+\infty}f(x)\mathrm{d}x$$若右端\textbf{两个积分都收敛}\footnote{两个积分必须都收敛,不能使用不存在+不存在=存在},则称反常积分收敛,否则称发散
                  \end{enumerate}
            \item \textbf{无界函数}的反常积分的概念与敛散性
                  \begin{defn}{瑕点的定义}{}
                      使$f(x)$在 $x_0$的邻域内无界的点即为瑕点,例如:$\lim_{x\to 0}\dfrac{1}{x}=\infty ,x=0$为函数的瑕点.
                  \end{defn}
                  设$F(x)$是$f(x)$在相应区间上的一个原函数,$x_0$为$f(x)$的瑕点.
                  \begin{enumerate}
                      \item 若 $x=a$ 是唯一瑕点,则
                            $$\int_a^bf(x)\mathrm{d}x=F(b)-\lim_{x\to a^+}F(x)$$
                            若上述极限存在,则称反常积分收敛,否则称发散. \footnote{前面定积分章节,要求$\int_a ^b f(x) dx$存在的必要条件是$f(x)$有界,但是此处似乎无界,也可使积分存在,似乎矛盾,需要注意的是,两者的积分不一样,前面要求有界的是黎曼积分,而此次可以无界的是反常积分,二者不同.}
                      \item 若 $x=b$ 是唯一瑕点,则
                            $$\int_a^bf(x)\mathrm{d}x=\lim_{x\to b^-}F(x)-F(a)$$
                            若上述极限存在,则称反常积分收敛,否则称发散.
                      \item 若 $x=c\in(a,b)$是唯一瑕点,则
                            $$\int_a^bf(x)\mathrm{d}x=\int_a^cf(x)\mathrm{d}x+\int_c^bf(x)\mathrm{d}x\:,$$
                            若右端两个积分都收敛,则称反常积分收敛,否则称发散.
                  \end{enumerate}
        \end{itemize}
    \end{defn}
    \begin{criterion}{敛散性的判定}{}
        \begin{itemize}
            \item 无穷区间
                  \begin{enumerate}
                      \item 比较判别法\footnote{可通过不等式放缩来证明}:设函数$f(x),g(x)$在区间$[a, +\infty )$上连续,并且$0\leqslant f(x)\leqslant g(x)(a\leqslant x<+\infty)$,则
                            \begin{itemize}
                                \item 当$\int_a^{+\infty}g(x) dx$收敛时,$\int_a^{+\infty} f(x) dx$收敛
                                \item 当$\int_a^{+\infty}f(x) dx$发散时,$\int_a^{+\infty} g(x) dx$发散
                            \end{itemize}
                      \item 比较判别法的极限形式\footnote{通过等价无穷小求极限来判断}:设函数$f(x),g(x)$在区间$[a,+\infty)$上连续,且$f(x)\geqslant0,g(x)>0,\lim_{x\to+\infty}\dfrac{f(x)}{g(x)}=\lambda$(有限或 $\infty)$,则
                            \begin{itemize}
                                \item 当$\lambda\neq0$且$\lambda\neq\infty$时$,\int_a^{+\infty} f(x)$d$x$与$\int_a^{+\infty} g(x)$d$x$有相同的敛散性
                                \item 当$\lambda=0$时,若$\int_{a}^{+\infty}g(x) dx$收敛,则$\int_a^{+\infty}f(x) dx$也收敛
                                \item 当$\lambda=\infty$时,若$\int_a^{+\infty} g(x)$d$x$发散,则$\int_a^{+\infty}f(x)$d$x$也发散
                            \end{itemize}
                  \end{enumerate}
            \item 无界函数
                  \begin{enumerate}
                      \item 比较判别法:设$f(x),g(x)$在$(a,b]$上连续,瑕点同为$x=a$,并且$0\leqslant f(x)\leqslant g(x)(a<x\leqslant b)$,则
                            \begin{itemize}
                                \item 当$\int_{a}^{b}g(x) dx$收敛时,$\int_a^{b}f(x) dx$收敛
                                \item 当$\int_{a}^{b}f(x) dx$ 发散时,$\int_a^{b}g(x) dx$发散
                            \end{itemize}
                      \item 比较判别法的极限形式:设$f(x),g(x)$在$(a,b]$上连续,瑕点同为$x=a$,并且$f(x)\geqslant0,g(x)>0(a<x\leqslant b),\lim_{x\to a^{+}}\dfrac{f(x)}{g(x)}=\lambda$(有限或 $\infty)$,则
                            \begin{itemize}
                                \item 当$\lambda\neq0$且$\lambda\neq\infty$时,$\int_{a}^{b}f(x)\mathrm{d}x$和$\int_{a}^{b}g(x)\mathrm{d}x$有相同的敛散性
                                \item 当$\lambda=0$时,若$\int_{a}^{b}g(x)\mathrm{d}x$收敛,则$\int_{a}^{b}f(x)\mathrm{d}x$也收敛
                                \item 当$\lambda=\infty$时,若$\int_{a}^{b}g(x)\mathrm{d}x$发散,则$\int_{a}^{b}f(x)\mathrm{d}x$也发散
                            \end{itemize}
                  \end{enumerate}
        \end{itemize}
    \end{criterion}
    \begin{conclusion}{两个重要结论}{}
        \begin{enumerate}
            \item $\int _0 ^1 \dfrac{1}{x^p} dx \begin{cases}
                          \text{收敛},0<p<1 \\
                          \text{发散},p \geq 1
                      \end{cases}$
                  \newline 由于$x\to 0$时,存在$\sin x \sim \cdots \sim x $,因此有$\int _0 ^1 \dfrac{1}{\sin x ^p} dx \begin{cases}
                          \text{收敛},0<p<1 \\
                          \text{发散},p \geq 1
                      \end{cases}$,凡是与$x$趋于0的“速度”一样的函数$f(x)$均可如上讨论
            \item $\int_{1}^{+\infty}\dfrac{1}{x^{p}}\mathrm{d}x\begin{cases}{\text{收敛,}p>1}\\{\text{发散,}p\leq1} \end{cases}$.如当$x\to+\infty$且$a>0$时,$ax+b$亦趋于$+\infty$,与$x$趋于$+\infty$的“速度”一样,当$ax+b\geq k>0$时,$\int_1^{+\infty}\dfrac1{(ax+b)^p} dx$依然满足$\begin{cases}\text{收敛,}p>1,\\\text{发散,}p\leq1.\end{cases}$
        \end{enumerate}
    \end{conclusion}
    \section{反常积分的计算}
    \begin{conclusion}{}{}
        //  TODO
    \end{conclusion}
    %  ############################ 正文部分
    \ifx\allfiles\undefined
\end{sloppypar}
\end{document}
\fi
