\ifx\allfiles\undefined
\documentclass[8pt a4paper, oneside, UTF8]{ctexbook} 
\input{../config/config}
\begin{document}
\begin{sloppypar}
    % \title{{\Huge{\textbf{考研高等数学笔记}}}}
\author{作者:LoafPhilosopher }
\date{\today}
\maketitle                   % 在单独的标题页上生成一个标题
\thispagestyle{empty}        % 前言页面不使用页码
% \begin{center}
% 	\Huge\textbf{前言}
% \end{center}
% 本笔记主要结合张宇老师和武忠祥老师的以及邂逅遗憾老师的内容进行编写
% \begin{flushright}
% 	\begin{tabular}{c}
% 		\today \newline 
% 	\end{tabular}
% \end{flushright}
\newpage                      % 新的一页
\pagestyle{plain}             % 设置页眉和页脚的排版方式(plain:页眉是空的,页脚只包含一个居中的页码)
\setcounter{page}{1}          % 重新定义页码从第一页开始
\pagenumbering{Roman}         % 使用大写的罗马数字作为页码

\begin{spacing}{1.5}
	\tableofcontents
\end{spacing}           % 生成目录
\newpage                      % 以下是正文
\pagestyle{plain}
\setcounter{page}{1}          % 使用阿拉伯数字作为页码
\pagenumbering{arabic}
% \setcounter{chapter}{-1}    % 设置 -1 可作为第零章绪论从第零章开始 
    \else
    \fi
    %  ############################ 正文部分
    \chapter{变限积分}
    \section{变限积分的概念}
    \begin{defn}{变限积分的定义}{}
        当$x$在$[a,b]$上变动时,对应于每一个$x$值,积分$\int_a^xf(t)dt$ 都有一个确定的值,因此$\int_a^xf(t)dt$ 是一个关于$x$的函数,记作
        $$
            F(x)=\int_{a}^{x}f(t)\mathrm{d}t(a\leqslant x\leqslant b)\:,
        $$
        称函数$F(x)$为\textbf{变上限的定积分}.同理可以定义变下限的定积分和上、下限都变化的定积分,这些都称为\textbf{变限积分},事实上,变限积分就是定积分的推广.
    \end{defn}
    \begin{criterion}{变限积分的性质}{}\footnote{此处应结合定积分存在定理进行理解}
        \begin{enumerate}
            \item 函数$f(x)$在$I$上可积,则函数$F(x)=\int_{a}^{x}f(t)dt$在$I$上连续\footnote{面积存在,那么面积连续}
            \item 函数$f(x)$在$I$上连续,则函数$F(x)=\int_{a}^{x}f(t)\mathrm{d}t$ 在$I$上可导且$F^{\prime}(x)=f(x)$.\footnote{函数连续,那么可以求面积,或者说连续函数必有原函数}
            \item $x=x_0\in I$是$f(x)$唯一的跳跃间断点,则$F(x)=\int_a^xf(t)dt$在$x_0$处不可导,且\newline$\begin{cases}F_{-}^{\prime}(x_{0})=\lim_{x\to x^-_{0}}f(x)\\F_{+}^{\prime}(x_{0})=\lim_{x\to x^+_{0}}f(x).\end{cases}$
            \item 若$x=x_0\in I$ 是$f(x)$唯一的可去间断点,则$F(x)=\int_a^xf(t)$d$t$在$x_0$处可导,且$F^\prime(x_0)=\lim_{x\to x_0}f(x).$
        \end{enumerate}
    \end{criterion}
    \section{变限积分的计算}
    \begin{defn}{变限积分计算公式}{}
        设 $F(x)=\int_{\varphi_{1}(x)}^{\varphi_{2}(x)}f(t)$d$t$,其中$f(x)$在$[a,b]$上连续,可导函数$\varphi_1(x)$ 和 $\varphi_{2}(x)$的值域在$[a,b]$上,则在函数 $\varphi_{1}(x)$和 $\varphi_{2}(x)$的公共定义域上,有
        $$F'(x)=\frac{\mathrm{d}}{\mathrm{d}x}\biggl[\int_{\varphi_{1}(x)}^{\varphi_{2}(x)}f(t)\mathrm{d}t\biggr]=f\bigl[\varphi_{2}(x)\bigr]\varphi_{2}^{\prime}\:(x)-f\bigl[\varphi_{1}(x)\bigr]\varphi_{1}^{\prime}\:(x)\:.$$
    \end{defn}
    \begin{conclusion}{变限积分与函数性质部分的结论}{}
        此处应参考以下章节:函数-函数的四种特性及重要结论-奇偶性\ref{jox2}。函数-函数的四种特性及重要结论-周期性\ref{zqxjl}。上述两个部分对变限积分的奇偶性和周期性进行了说明。
    \end{conclusion}
    %  ############################ 正文部分
    \ifx\allfiles\undefined
\end{sloppypar}
\end{document}
\fi
