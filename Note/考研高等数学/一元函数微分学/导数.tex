\ifx\allfiles\undefined
\documentclass[12pt, a4paper, oneside, UTF8]{ctexbook}  % +  这一句是新增加的
\usepackage{amsmath}   % 数学公式
\usepackage[dvipsnames]{xcolor}
\usepackage{amsthm}    % 定理环境
\usepackage{amssymb}   % 更多公式符号
\usepackage{graphicx}  % 插图
\usepackage{mathrsfs}  % 数学字体
\usepackage{enumitem}  % 列表
\usepackage{geometry}  % 页面调整
\usepackage{unicode-math}
\usepackage{extarrows}
\usepackage{subfigure}
\usepackage{extarrows}
\usepackage{footnote}
\usepackage{svg}
\usepackage[colorlinks,linkcolor=black]{hyperref}
\usepackage{supertabular}
\usepackage{tcolorbox}
\usepackage{ulem}
\usepackage{framed}
\usepackage{float}
\usepackage{microtype}
\newcommand{\arccot}{\mathrm{arccot}\,}
\tcbuselibrary{breakable}
\tcbuselibrary{most}
\newcounter{problemname}
\newenvironment{solution}{\par\noindent\textbf{解答. }}{\par}
\newenvironment{note}{\par\noindent\textbf{题目\arabic{problemname}的注记. }}{\par}
\definecolor{shadecolor}{RGB}{241, 241, 255}
\newenvironment{problem}{\begin{shaded}\stepcounter{problemname}\par\noindent\textbf{题目\arabic{problemname}. }}{\end{shaded}\par}

\graphicspath{ {figure/},{../figure/}, {config/}, {../config/} }  % 配置图形文件检索目录
\linespread{1.2} % 行高

% 页码设置
\geometry{top=25.4mm,bottom=25.4mm,left=20mm,right=20mm,headheight=2.17cm,headsep=4mm,footskip=12mm}

% 设置列表环境的上下间距
\setenumerate[1]{itemsep=5pt,partopsep=0pt,parsep=\parskip,topsep=5pt}
\setitemize[1]{itemsep=5pt,partopsep=0pt,parsep=\parskip,topsep=5pt}
\setdescription{itemsep=5pt,partopsep=0pt,parsep=\parskip,topsep=5pt}

% 定理环境
% ########## 定理环境 start ####################################

% #### 将 config.tex 中的定理环境的对应部分替换为如下内容
% 定义单独编号,其他四个共用一个编号计数 这里只列举了五种,其他可类似定义(未定义的使用原来的也可)
\newtcbtheorem[auto counter, number within=section, list type=subsubsection, list inside=toc]{defn}{定义}
{
    colback=green!5,colframe=green!35!black,fonttitle=\bfseries, title={Comment \thetcbcounter}, list entry={Comment \thetcbcounter\quad}, %标题
    breakable, %支持跨页
    before upper={\parindent10pt\noindent},  % 支持缩进。\noindent:首行不缩进
    % left = 2mm, %文字离线框左边的边距
    % right = 1mm,%同上
    % top = 1mm,%同上
    % bottom = 1mm,%同上
    % arc is angular = 1mm, % 棱角线框
    % sharp corners, % 直角线框
    % enhanced,frame hidden, % 隐藏线框
    % enhanced, drop fuzzy shadow,  % 显示阴影
}
{def}

\newtcbtheorem[auto counter, number within=section, list type=subsubsection, list inside=toc]{lemma}{引理}
{
    colback=SeaGreen!10!CornflowerBlue!10,colframe=RoyalPurple!55!Aquamarine!100!,fonttitle=\bfseries, title={Comment \thetcbcounter}, list entry={Comment \thetcbcounter\quad}, %标题
    breakable, %支持跨页
    before upper={\parindent10pt\noindent},  % 支持缩进。\noindent:首行不缩进
    % left = 2mm, %文字离线框左边的边距
    % right = 1mm,%同上
    % top = 1mm,%同上
    % bottom = 1mm,%同上
    % arc is angular = 1mm, % 棱角线框
    % sharp corners, % 直角线框
    % enhanced,frame hidden, % 隐藏线框
    % enhanced, drop fuzzy shadow,  % 显示阴影
}
{lem}


\newtcbtheorem[auto counter, number within=section, list type=subsubsection, list inside=toc]{them}{定理}
{
    colback=Salmon!20, colframe=Salmon!90!Black,fonttitle=\bfseries, title={Comment \thetcbcounter}, list entry={Comment \thetcbcounter\quad}, %标题
    breakable, %支持跨页
    before upper={\parindent10pt\noindent},  % 支持缩进。\noindent:首行不缩进
    % left = 2mm, %文字离线框左边的边距
    % right = 1mm,%同上
    % top = 1mm,%同上
    % bottom = 1mm,%同上
    % arc is angular = 1mm, % 棱角线框
    % sharp corners, % 直角线框
    % enhanced,frame hidden, % 隐藏线框
    % enhanced, drop fuzzy shadow,  % 显示阴影
}
{them}
\newtcbtheorem[auto counter, number within=section, list type=subsubsection, list inside=toc]{criterion}{注}
{
    colback=CornflowerBlue!10,colframe=RoyalPurple!55!Aquamarine!100!,fonttitle=\bfseries, title={Comment \thetcbcounter}, list entry={Comment \thetcbcounter\quad}, %标题
    breakable, %支持跨页
    before upper={\parindent10pt\noindent},  % 支持缩进。\noindent:首行不缩进
    % left = 2mm, %文字离线框左边的边距
    % right = 1mm,%同上
    % top = 1mm,%同上
    % bottom = 1mm,%同上
    % arc is angular = 1mm, % 棱角线框
    % sharp corners, % 直角线框
    % enhanced,frame hidden, % 隐藏线框
    % enhanced, drop fuzzy shadow,  % 显示阴影
}
{cri}

\newtcbtheorem[auto counter, number within=section, list type=subsubsection, list inside=toc]{corollary}{推论}
{
    colback=Emerald!10,colframe=cyan!40!black,fonttitle=\bfseries, title={Comment \thetcbcounter}, list entry={Comment \thetcbcounter\quad}, %标题
    breakable, %支持跨页
    before upper={\parindent10pt\noindent},  % 支持缩进。\noindent:首行不缩进
    % left = 2mm, %文字离线框左边的边距
    % right = 1mm,%同上
    % top = 1mm,%同上
    % bottom = 1mm,%同上
    % arc is angular = 1mm, % 棱角线框
    % sharp corners, % 直角线框
    % enhanced,frame hidden, % 隐藏线框
    % enhanced, drop fuzzy shadow,  % 显示阴影
}
{cor}
% colback=red!5,colframe=red!75!black

% ######### 定理环境 end  #####################################

% ↓↓↓↓↓↓↓↓↓↓↓↓↓↓↓↓↓ 以下是自定义的命令  ↓↓↓↓↓↓↓↓↓↓↓↓↓↓↓↓

% 用于调整表格的高度  使用 \hline\xrowht{25pt}
\newcommand{\xrowht}[2][0]{\addstackgap[.5\dimexpr#2\relax]{\vphantom{#1}}}

% 表格环境内长内容换行  
\newcommand{\tabincell}[2]{\begin{tabular}{@{}#1@{}}#2\end{tabular}}

% 使用\linespread{1.5} 之后 cases 环境的行高也会改变,重新定义一个 ca 环境可以自动控制 cases 环境行高
\newenvironment{ca}[1][1]{\linespread{#1} \selectfont \begin{cases}}{\end{cases}}
% 和上面一样
\newenvironment{vx}[1][1]{\linespread{#1} \selectfont \begin{vmatrix}}{\end{vmatrix}}

\def\d{\textup{d}} % 直立体 d 用于微分符号 dx
\def\R{\mathbb{R}} % 实数域
\newcommand{\bs}[1]{\boldsymbol{#1}}    % 加粗,常用于向量
\newcommand{\ora}[1]{\overrightarrow{#1}} % 向量

% 数学 平行 符号
\newcommand{\pll}{\kern 0.5em/\kern -0.8em /\kern 0.5em}

% 用于空行\myspace{1} 表示空一行 填 2 表示空两行  
\newcommand{\myspace}[1]{\par\vspace{#1\baselineskip}}

\begin{document}
\begin{sloppypar}

    % \title{{\Huge{\textbf{高等数学笔记}}}}
\author{作者:于家崇}
\date{\today}
\maketitle                   % 在单独的标题页上生成一个标题

\thispagestyle{empty}        % 前言页面不使用页码
\begin{center}
	\Huge\textbf{前言}
\end{center}

If a job is worth doing,it's worth doing well
\begin{flushright}
	\begin{tabular}{c}
		\today \\ 如果一件事值得去做,那就值得去做好
	\end{tabular}
\end{flushright}

\newpage                      % 新的一页
\pagestyle{plain}             % 设置页眉和页脚的排版方式(plain:页眉是空的,页脚只包含一个居中的页码)
\setcounter{page}{1}          % 重新定义页码从第一页开始
\pagenumbering{Roman}         % 使用大写的罗马数字作为页码
\tableofcontents              % 生成目录

\newpage                      % 以下是正文
\pagestyle{plain}
\setcounter{page}{1}          % 使用阿拉伯数字作为页码
\pagenumbering{arabic}
% \setcounter{chapter}{-1}    % 设置 -1 可作为第零章绪论从第零章开始 
    \else
    \fi
    %  ############################ 正文部分
    \chapter{导数}

    \section{导数的定义}
    \begin{defn}{导数的定义}{}
        设函数$y=f(x)$在点 $x_0$的某个邻域内有定义,当自变量 $x$ 在 $x_0$处取得增量 $\Delta x$(点$x_0+\Delta x$仍在该邻域内)时,相应地,因变量取得增量$\Delta y=f(x_0+\Delta x)-f(x_0)$;如果$\Delta  y$与$\Delta x$之比当 $\Delta x \to 0$时的极限存在,那么称函数 $y=f(x)$在点 $x_0$处可导,并称这个极限为函数 $y=f(x)$在点 $x_0$处的导数,记为$f^{\prime}(x_0)$,即
        $$
            f^{'}(x_{0})=\lim_{x\to x_{0}}\frac{f(x)-f(x_{0})}{x-x_{0}}=\lim_{\Delta x\to0}\frac{\Delta y}{\Delta x}=\lim_{\Delta x\to0}\frac{f(x_{0}+\Delta x)-f(x_{0})}{\Delta x}
        $$
        也可记作$y^{\prime}\mid_{x=x_{0}},\left.\frac{\mathrm{d}y}{\mathrm{d}x}\right|_{x=x_{0}}\text{或}\frac{\mathrm{d}f(x)}{\mathrm{d}x}|_{x=x_{0}}.$
    \end{defn}
    \begin{criterion}{}{}
        在考题中,增量$\Delta x$一般会被命题人广义化为"$\square$",即
        $$
            f'(x_0)=\lim_{\Delta x \to 0}\frac{f(x_0+\Delta x)-f(x_0)}{\Delta x}\xlongequal{\text{广义化}} \lim_{ \square \to 0}\frac{f(x_0 + \square)-f(x_0)}{\square}
        $$
    \end{criterion}
    \subsection{导数与导函数}
    \begin{center}
        导数是一个极限,导函数是一个函数
    \end{center}

    \subsection{单侧导数}
    \begin{defn}{}{}
        \begin{center}
            函数$f(x)$在$x_0$点可导的充分必要条件是左导数和右导数存在且相等,其表达式为
            $$
                \begin{aligned}&\lim_{h \to0^-}\frac{f(x_0+h )-f(x_0)}{h }\overset{\text{记}}{=}f'_{-}\left(x_0\right),\\\\&\lim_{h \to0^+}\frac{f(x_0+h )-f(x_0)}{h }\overset{\text{记}}{=}f'_{+}\left(x_0\right),\end{aligned}
            $$
        \end{center}
    \end{defn}
    \section{导数的计算}

    \subsection{基本求导公式}
    \begin{center}
        \boxed{
            $$
                \begin{aligned}
                     & \left(1\right)\left(C\right)^{\prime}=0;                        &  & \left(2\right)\left(x^{\alpha}\right)^{\prime}=\alpha x^{\alpha-1}; \\
                     & \left(3\right)\left(a^{x}\right)^{\prime}=a^{x}\ln a;           &  & \left(4\right)\left(\mathrm{e}^{x}\right)^{\prime}=\mathrm{e}^{x};  \\
                     & \left(5\right)\left(\log_{a}x\right)^{\prime}=\frac{1}{x\ln a}; &  & (6)(\ln|x|)^{\prime}=\frac{1}{x};                                   \\
                     & \left(7\right)\left(\sin x\right)^{\prime}=\cos x;              &  & \left(8\right)\left(\cos x\right)^{\prime}=-\sin x;                 \\
                     & \left(9\right)\left(\tan x\right)^{\prime}=\sec^{2}x;           &  & \left(10\right)\left(\cot x\right)^{\prime}=-\csc^{2}x;             \\
                     & (11)(\sec x)'=\sec x\tan x;                                     &  & (12)(\csc x)'=-\csc x\cot x;                                        \\
                     & (13)(\arcsin x)'=\frac{1}{\sqrt{1-x^2}};                        &  & (14)(\arccos x)'=-\frac{1}{\sqrt{1-x^2}};                           \\
                     & (15)(\arctan x)'=\frac{1}{1+x^2};                               &  & (16)(\operatorname{arccot}x)'=-\frac{1}{1+x^2}.
                \end{aligned}
            $$
        }
    \end{center}
    \subsection{有理运算法则}
    设$u=u(x),v=v(x)$在$x$处可导,则
    $$
        (u\pm v)^{\prime}=u^{\prime}\pm v^{\prime}
    $$

    $$
        \left(uv\right)^{\prime}=u^{\prime}v+uv^{\prime}
    $$

    $$
        \left(\frac uv\right)^{\prime}=\frac{u^{\prime}v-uv^{\prime}}{v^2}\quad(v\neq0)
    $$

    \subsection{分段函数的导数}
    \textbf{分段函数在分段点处的导数,一定要要用定义来求,结果有可能是不可导的}

    即:设$f(x)=\begin{cases}f_1(x),&x\geqslant x_0,\\f_2(x),&x<x_0,\\\end{cases}$,则在$x_0$处用导数定义求导:$\lim _ { x \rightarrow x _ { 0 }^{-} } \frac { f ( x ) - f ( x _ { 0 } ) } { x - x _ { 0 } }, \lim _ { x \rightarrow x_0^+ } \frac { f ( x ) - f ( x _0)  } { x - x _ { 0 } }$.最后还需要根据左右导数是否相等来判定分段点$f'(x_0)$

    \subsection{复合函数的导数与微分形式不变性}
    \subsubsection{复合导数}
    \begin{defn}{复合函数导数的定义}{}
        设$y = f ( g ( x ) )$是由$y=f(z)$,$z=g(x)$复合而成,且$f(z)$,$g(x)$均可导,则$\frac { d y } { d x } = \frac { d y } { d z } \cdot \frac { d z } { d x } = f ^ { \prime } ( z ) \cdot g ^ { \prime } ( x ) = f ^ { \prime } ( g ( x )) \cdot g ^ { \prime } ( x )$
    \end{defn}
    \subsubsection{微分形式不变形}
    指无论$u$是中间变量还是自变量,$dy=f'(u)du$
    \subsection{反函数的导数}

    \begin{defn}{}{}
        反函数的导数等于原函数导数的倒数,即
        $$
            \frac { d y } { d x } = \frac { 1 } { \frac { d x } { d y } }
        $$
    \end{defn}
    \begin{criterion}{解释}{}
        \begin{table}[H]
            \begin{center}
                \begin{tabular}{|l|l|l|}
                    \hline
                    函数类型                                            & 自变量与因变量                                               & 函数表达式             \\ \hline
                    \begin{tabular}[c]{@{}l@{}}直接\\ 函数\end{tabular} & \begin{tabular}[c]{@{}l@{}}自变量:x\\ 因变量:y\end{tabular} & $y=\frac{1}{6} x$ \\ \hline
                    反函数                                             & \begin{tabular}[c]{@{}l@{}}自变量:y\\ 因变量:x\end{tabular} & $x=6y$            \\ \hline
                    反函数                                             & \begin{tabular}[c]{@{}l@{}}自变量:x\\ 因变量:y\end{tabular} & $y=6x$            \\ \hline
                \end{tabular}
            \end{center}
        \end{table}
    \end{criterion}
    需要注意的是,定理二中采用的是解释中第二行的形式
    \newline
    \newline
    反函数的二阶导函数为:$y_{xx}''=\frac{\operatorname{d}^2y}{\operatorname{d}x^2}=\frac{\operatorname{d}\left(\frac{\operatorname{d}y}{\operatorname{d}x}\right)}{\operatorname{d}x}=\frac{\operatorname{d}\left(\frac{1}{x_y'}\right)}{\operatorname{d}x}=\frac{\operatorname{d}\left(\frac{1}{x_y'}\right)}{\operatorname{d}y}\cdot\frac{1}{x_y'}=\frac{-x_{yy}''}{(x_y')^3}$
    \subsection{参数方程所确定的函数的导数}
    \begin{defn}{}{}
        设$y=f(x)$的参数方程是$\left\{\begin{aligned}x&=\varphi(t),\\y&=\psi (t),\end{aligned}\right.(\alpha<t<\beta)$确定的函数

        如果$\varphi(t)$和$\psi(t)$都可导,且$\varphi'(t) \neq 0$则其一阶导可写为
        $$
            \frac{\mathrm{d}y}{\mathrm{d}x}=\frac{\psi^{\prime}(t)}{\varphi^{\prime}(t)}
        $$
        若$\varphi(t)$ 和$\psi(t)$ 二阶可导,且 $\varphi^{\prime}(t)\neq 0$,则
        $$
            \begin{aligned}
                \frac{\operatorname{d}^2y}{\operatorname{d}x^2} & =\frac{d}{dx}(\frac{dy}{dx}) =\frac{\operatorname{d}}{\operatorname{d}t}\Big(\frac{\psi'(t)}{\varphi'(t)}\Big) \times \frac{1}{\varphi'(t)} \\
                                                                & =\frac{\psi''(t)\varphi'(t)-\varphi''(t)\psi'(t)}{\varphi'^2(t)}\times \frac{1}{\varphi'(t)}                                                \\
                                                                & =\frac{\psi''(t)\varphi'(t)-\varphi''(t)\psi'(t)}{\varphi'^3(t)}
            \end{aligned}
        $$
    \end{defn}
    \subsection{对数函数求导法}
    对于多项相乘,相除,开方,乘方的式子,一般对其先求对数在求导数.即
    $$
        \ln y=\ln f(x)
    $$
    然后两边对$x$求导即可.
    \subsection{幂指函数求导法}
    对于$u(x)^{v(x)}$函数,可采用$e^{v(x)\ln u(x)}$进行转换求导
    \subsection{变上限积分}

    \section{导数的几何意义}
    \begin{defn}{}{}
        $y=f(x)$在$x_0$处导数是$f(x)$在$x_0$处切线的斜率$k_\text{切} = f ^ { \prime } ( x _ { 0 } )$并且$k_\text{切}*k_\text{法}=-1$
        \newline
        在($x_0$,$y_0$)处,切线:$$y - y _ { 0 } = f ^ { \prime } ( x _ { 0 } ) ( x - x _ { 0 } )$$
        法线:$$y - y _ { 0 } = - \frac { 1 } { f ^ { \prime } ( x _ { 0 } ) } ( x - x _ { 0 } )$$
    \end{defn}
    \section{函数可导性与连续性的关系}
    \begin{center}
        \boxed{
            $$
                \text{可导一定连续,连续不一定可导}.
            $$
        }
    \end{center}
    \begin{criterion}{关于函数可导性与联系性的解释}{}
        函数在$x_0$处可导的充分必要条件是左导数和右导数均存在且相等.但是通过对函数$|x|$的分析,可以知道的是函数在原点$O$处左右导数均存在,但是却不相等.
    \end{criterion}
    %  ############################ 正文部分
    \section{高阶导数}
    \begin{defn}{高阶导数的定义}{}
        函数$y=f(x)$具有$n$阶导数,也常说成函数$f(x)$为$n$阶可导,如果函数$f(x)$在点$x$处具有$n$阶导数,那么$f(x)$在点$x$的某一邻域内必定具有一切低于$n$阶的导数.二阶及二阶以上的导数统称为高阶导数.
        记作:
        $$
            f ^ { ( n ) } ( x ) = \frac { d ^ { ( n ) } f  { ( x ) } } { d x ^ { ( n ) } }
        $$

    \end{defn}
    \begin{itemize}
        \item 找规律+归纳
              $( \frac { 1 } { x + 1 } ) ^ { ( n ) } = ( - 1 ) ^ { n } ( x + 1 ) ^ { - ( n + 1 ) } \cdot n !$
        \item 莱布尼兹公式:适用于两个函数相乘求高阶导数$( u v ) ^ { ( n ) } = \sum _ { k = 0 } ^ { n } C _ { n } ^ { k } u ^ { ( n - k ) } \cdot v ^ { ( k ) }$
    \end{itemize}
    \section{隐函数}
    \begin{defn}{隐函数与显函数的定义}{}
        \begin{itemize}
            \item 隐函数:$y$与$x$的关系隐含在一个等式中,$F(x,y)=0$,eg:$x^2+y^2=4$
            \item  显函数:因变量与自变量在等式两端,$y$和$x$各占一边,eg:$y=3x$​
        \end{itemize}
    \end{defn}
    \subsection{隐函数求导法则}
    \begin{defn}{}{}
        y看作与x相关的量,等式两端同时对x求导.如$y \rightarrow y ^ { \prime } ( x )  \text{或} y ^ { 2 } \rightarrow 2 y ( x ) \cdot y ^ { \prime } ( x )   \text{或} ln y  = \frac { 1 } { y ( x ) } \cdot y ^ { \prime } ( x )$
    \end{defn}
    \begin{problem}
    求$y=x^{\sin x}$导数
    \end{problem}
    \begin{solution}
        对等式两边取对数可得:$\ln y=\sin x \ln x$\\
        根据隐函数求导法则,对等式两边求导可得:$\frac{y'}{y}=\cos x \times \ln x +\frac{\sin x}{x}$\\
        化简可得导数为$y'=x^{\sin x} \times( \cos x \times \ln x +\frac{\sin x}{x})$
    \end{solution}
    \begin{problem}
    求$y=\sqrt{\frac{\left(x-1\right)\left(x-2\right)}{\left(x-3\right)\left(x-4\right)}}$导数
    \end{problem}
    \begin{solution}
        对等式两边取对数可得:$\ln y=\frac{1}{2} \times \ln \frac{(x-1)(x-2)}{(x-3)(x-4)}$\\
        即:$2\ln y=\ln (x-1) +\ln (x-2) -\ln (x-3)-\ln (x-4)$\\
        根据隐函数求导法则,对等式两边求导可得:$2 \times \frac{y'}{y}=\frac{1}{x-1}+\frac{1}{x-2}-\frac{1}{x-3}-\frac{1}{x-4}$\\
        化简可得导数为$y'=\sqrt{\frac{\left(x-1\right)\left(x-2\right)}{\left(x-3\right)\left(x-4\right)}} \times ({\frac{1}{x-1}+\frac{1}{x-2}-\frac{1}{x-3}-\frac{1}{x-4}})$
    \end{solution}
    \ifx\allfiles\undefined
\end{sloppypar}
\end{document}
\fi
