\ifx\allfiles\undefined
\documentclass[10pt, a4paper, oneside, UTF8]{ctexbook} 
\input{../config/config}
\begin{document}
\begin{sloppypar}
    % \title{{\Huge{\textbf{考研高等数学笔记}}}}
\author{作者:LoafPhilosopher }
\date{\today}
\maketitle                   % 在单独的标题页上生成一个标题
\thispagestyle{empty}        % 前言页面不使用页码
% \begin{center}
% 	\Huge\textbf{前言}
% \end{center}
% 本笔记主要结合张宇老师和武忠祥老师的以及邂逅遗憾老师的内容进行编写
% \begin{flushright}
% 	\begin{tabular}{c}
% 		\today \newline 
% 	\end{tabular}
% \end{flushright}
\newpage                      % 新的一页
\pagestyle{plain}             % 设置页眉和页脚的排版方式(plain:页眉是空的,页脚只包含一个居中的页码)
\setcounter{page}{1}          % 重新定义页码从第一页开始
\pagenumbering{Roman}         % 使用大写的罗马数字作为页码

\begin{spacing}{1.5}
	\tableofcontents
\end{spacing}           % 生成目录
\newpage                      % 以下是正文
\pagestyle{plain}
\setcounter{page}{1}          % 使用阿拉伯数字作为页码
\pagenumbering{arabic}
% \setcounter{chapter}{-1}    % 设置 -1 可作为第零章绪论从第零章开始 
    \else
    \fi
    %  ############################ 正文部分
    \chapter{连续}
    \section{函数的连续性}
    \begin{defn}{连续点的定义}{}
        设函数$y=f(x)$在点$x_0$的某一邻域内有定义,如果
        $$
            \lim_{x\to x_0}f(x)=f(x_0)
        $$
        那就称为函数$y=f(x)$在点$x_0$连续.
    \end{defn}
    \begin{criterion}{}{}
        \begin{itemize}
            \item 当极限需要讨论时:$\lim_{x\to x_0^+}f\left(x\right)=\lim_{x\to x_0^-}f\left(x\right)=f\left(x_{0}\right)\Leftrightarrow f\left(x\right)$在点$x_{0}$处连续
            \item 连续性的四则运算:设$f(x)$与$g(x)$都在点$x=x_0$处连续,则$f(x)\pm g(x)$与$f(x)g(x)$在点$x=x_{0}$处连续,当$g(x_0)\neq0$ 时,$f(x)/g(x)$在点$x=x_{0}$处也连续。
            \item 复合函数的连续性:设$u=\varphi(x)$在点$x=x_0$处连续,$y=f(u)$在点$u=u_0$处连续,且$u_{0}=\varphi(x_{0})$,则$f\left[\varphi(x)\right]$在点$x=x_{0}$处连续。
            \item 反函数的连续性:设$y=f(x)$在区间$I_x$上单调且连续,则反函数$x=\varphi(y)$在对应的区间 $I_{y}=\{y|y=f(x),x\in I_{x}\}$ 上连续且有相同的单调性
            \item \textbf{$f(x)$在点$x=x_0$处连续,且$f(x_0)>0$(或$f(x_0)<0$),则存在$\delta>0$,使得当$|x-x_0|<\delta$时 $f\left(x\right)>0\left(\text{或}f\left(x\right)<0\right).$ }
        \end{itemize}
    \end{criterion}
    \section{函数的间断点}
    \subsection{间断点的相关概念}
    \begin{itemize}
        \item 可去间断点:若$\lim_{x\to x_0}f(x)=A\neq f(x_0)(f(x_0)$甚至可以无定义),则这类间断点称为可去间断点
              \begin{figure}[H]
                  \centering \includegraphics[width=
                      0.4 \linewidth]{3.2.1.eps} \caption{可去间断点函数图像}
              \end{figure}
        \item 跳跃间断点\footnote{一点极限存在\neq f(x)在$x_0$连续}:若$\lim_{x\to x_0^-}f(x)$与$\lim_{x\to x_0^+}f(x)$都存在,但$\lim_{x\to x_0^+}f(x)\neq\lim_{x\to x_0^-}f(x)$,则这类间断点称为跳跃间断点
              \begin{figure}[H]
                  \centering \includegraphics[width=
                      0.4 \linewidth]{3.2.2.eps} \caption{跳跃间断点函数图像}
              \end{figure}
        \item 无穷间断点:若$\lim_{x\to x_0}f(x)=\infty$,则这类间断点称为无穷间断点,如$y=\tan x$
              \begin{figure}[H]
                  \centering \includegraphics[width=
                      0.4 \linewidth]{1.3.6.eps} \caption{无穷间断点函数tan图像}
              \end{figure}
        \item 振荡间断点:若$\lim_{x\to x_0}f(x)$振荡不存在,则这类间断点称为振荡间断点
              \begin{figure}[H]
                  \centering \includegraphics[width=
                      0.4 \linewidth]{3.2.3.eps} \caption{振荡间断点函数$\sin \frac{1}{x}$图像}
              \end{figure}
    \end{itemize}
    \subsection{间断点的分类}
    通过求函数在该点的左右极限来判断
    \begin{itemize}
        \item 第一类间断点:$\lim _ { x \rightarrow x _ { 0 } ^{-}} f ( x )$​ 和$\lim _ { x \rightarrow x _ { 0 }^ {+}} f ( x )$​ 均存在
              \begin{itemize}
                  \item 可去\footnote{可去间断点上极限存在但是导数不存在}:$\lim _ { x \rightarrow x_0 ^ { - } } f ( x ) = \lim _ { x \rightarrow x_0^{+} } f  ( x ) \neq f(x_0)$
                  \item 跳跃:$\lim _ { x \rightarrow x_0^{-} } f ( x ) \not= \lim _ { x \rightarrow x_0^{+} } f ( x )$
              \end{itemize}
        \item 第二类间断点:除第一类以外的间断点$\implies \lim _ { x \rightarrow x _ { 0 } ^{-}} f ( x )$和$\lim _ { x \rightarrow x _ { 0 }^ {+}} f ( x )$​ 均至少一个不存在
    \end{itemize}
    %  ############################ 正文部分
    \ifx\allfiles\undefined
\end{sloppypar}
\end{document}
\fi
