\ifx\allfiles\undefined
\documentclass[8pt a4paper, oneside, UTF8]{ctexbook} 
\input{../config/config}
\begin{document}
\begin{sloppypar}
    % \title{{\Huge{\textbf{考研高等数学笔记}}}}
\author{作者:LoafPhilosopher }
\date{\today}
\maketitle                   % 在单独的标题页上生成一个标题
\thispagestyle{empty}        % 前言页面不使用页码
% \begin{center}
% 	\Huge\textbf{前言}
% \end{center}
% 本笔记主要结合张宇老师和武忠祥老师的以及邂逅遗憾老师的内容进行编写
% \begin{flushright}
% 	\begin{tabular}{c}
% 		\today \newline 
% 	\end{tabular}
% \end{flushright}
\newpage                      % 新的一页
\pagestyle{plain}             % 设置页眉和页脚的排版方式(plain:页眉是空的,页脚只包含一个居中的页码)
\setcounter{page}{1}          % 重新定义页码从第一页开始
\pagenumbering{Roman}         % 使用大写的罗马数字作为页码

\begin{spacing}{1.5}
	\tableofcontents
\end{spacing}           % 生成目录
\newpage                      % 以下是正文
\pagestyle{plain}
\setcounter{page}{1}          % 使用阿拉伯数字作为页码
\pagenumbering{arabic}
% \setcounter{chapter}{-1}    % 设置 -1 可作为第零章绪论从第零章开始 
    \else
    \fi
    %  ############################ 正文部分
    \chapter{连续}
    \section{函数的连续性}
    \begin{defn}{连续点的定义}{}
        设函数$y=f(x)$在点$x_0$\textbf{的某一邻域内有定义},如果
        $$
            \lim_{x\to x_0}f(x)=f(x_0)
        $$
        那就称为函数$y=f(x)$在点$x_0$连续.
    \end{defn}
    \begin{criterion}{函数连续的性质}{}
        \begin{itemize}
            \item 当极限需要讨论时:
            $$
                \lim_{x\to x_0^+}f\left(x\right)=\lim_{x\to x_0^-}f\left(x\right)=f\left(x_0\right)\Leftrightarrow f\left(x\right)\text{ 在点 }x_0\text{ 处连续}
            $$
            \item \textbf{一点连续不能推出邻域连续}:以函数$f(x)=x D(x)$为例,其中$D(x)$为狄利克雷函数:该函数在$x=0$时极限为0,函数值也为0,因此函数在$x=0$点连续,但是其邻域内所有点都不连续.
            \item 连续性的四则运算:设$f(x)$与$g(x)$都在点$x=x_0$处连续,则$f(x)\pm g(x)$与$f(x)g(x)$在点$x=x_{0}$处连续,当$g(x_0)\neq0$ 时,$f(x)/g(x)$在点$x=x_{0}$处也连续。
            \item 复合函数的连续性:设$u=\varphi(x)$在点$x=x_0$处连续,$y=f(u)$在点$u=u_0$处连续,且$u_{0}=\varphi(x_{0})$,则$f\left[\varphi(x)\right]$在点$x=x_{0}$处连续。
            \item 反函数的连续性:设$y=f(x)$在区间$I_x$上单调且连续,则反函数$x=\varphi(y)$在对应的区间 $I_{y}=\{y|y=f(x),x\in I_{x}\}$ 上连续且有相同的单调性
            \item \textbf{$f(x)$在点$x=x_0$处连续,且$f(x_0)>0$(或$f(x_0)<0$),则存在$\delta>0$,使得当$|x-x_0|<\delta$时 $f\left(x\right)>0\left(\text{或}f\left(x\right)<0\right).$ }
        \end{itemize}
    \end{criterion}
    \section{函数的间断点}
    \subsection{间断点的相关概念}
    \textcolor{red}{讨论间断点的前提:函数$f(x)$在点$x_0$的某去心领域内有定义}
    \begin{defn}{可去间断点的定义}{}
        可去间断点:若$\lim_{x\to x_0}f(x)=A\neq f(x_0)(f(x_0)$甚至可以无定义),则这类间断点称为可去间断点
        \begin{center}
            \tikzset{every picture/.style={line width=0.5pt}} %set default line width to 0.75pt        
            \begin{tikzpicture}[x=0.5pt,y=0.5pt,yscale=-1,xscale=1]
            \draw  (151,148.43) -- (349.86,148.43)(250.09,67.87) -- (250.09,231.3) (342.86,143.43) -- (349.86,148.43) -- (342.86,153.43) (245.09,74.87) -- (250.09,67.87) -- (255.09,74.87) (270.09,143.43) -- (270.09,153.43)(290.09,143.43) -- (290.09,153.43)(310.09,143.43) -- (310.09,153.43)(330.09,143.43) -- (330.09,153.43)(230.09,143.43) -- (230.09,153.43)(210.09,143.43) -- (210.09,153.43)(190.09,143.43) -- (190.09,153.43)(170.09,143.43) -- (170.09,153.43)(245.09,128.43) -- (255.09,128.43)(245.09,108.43) -- (255.09,108.43)(245.09,88.43) -- (255.09,88.43)(245.09,168.43) -- (255.09,168.43)(245.09,188.43) -- (255.09,188.43)(245.09,208.43) -- (255.09,208.43) ;
            \draw   ;
            \draw    (250.09,148.43) -- (313.36,62.43) ;
            \draw   (297,81.43) .. controls (297,80.13) and (298.06,79.07) .. (299.36,79.07) .. controls (300.66,79.07) and (301.71,80.13) .. (301.71,81.43) .. controls (301.71,82.73) and (300.66,83.79) .. (299.36,83.79) .. controls (298.06,83.79) and (297,82.73) .. (297,81.43) -- cycle ;
            \draw  [fill={rgb, 255:red, 0; green, 0; blue, 0 }  ,fill opacity=1 ] (298.64,147.57) .. controls (298.64,146.55) and (299.47,145.71) .. (300.5,145.71) .. controls (301.53,145.71) and (302.36,146.55) .. (302.36,147.57) .. controls (302.36,148.6) and (301.53,149.43) .. (300.5,149.43) .. controls (299.47,149.43) and (298.64,148.6) .. (298.64,147.57) -- cycle ;
            \draw (246,43) node [anchor=north west][inner sep=0.75pt]   [align=left] {$\displaystyle y$};
            \draw (353,138) node [anchor=north west][inner sep=0.75pt]   [align=left] {$\displaystyle x$};
            \draw (233,149) node [anchor=north west][inner sep=0.75pt]   [align=left] {$\displaystyle O$};
            \draw (122,138) node [anchor=north west][inner sep=0.75pt]   [align=left] {$ $};
            \draw (181,236) node [anchor=north west][inner sep=0.75pt]   [align=left] {可去间断点函数图像};
            \end{tikzpicture}            
        \end{center}
    \end{defn}
    \begin{problem}
        函数 $f\left(x\right)=\dfrac{(x+1)|x-1|}{e^{\frac{1}{x-2}}\ln|x|}$的可去间断点的个数为
    \end{problem}
    \be gin{solution}
        该题中可疑点为$x = \pm 1,2,0$,对上述四点求极限可得:
        $\lim_{x\to 0}=0$,但是函数$f(x)$在$x=0$点无定义.因此$x=0$是可去间断点.$\lim_{x\to 1}f(x)$时$\lim_{x\to1^+}\neq \lim_{x\to 1^-}$.因此$x=1$是跳跃间断点.$\lim_{x\to -1}=-2\sqrt[3]{e}$,因此$x=-1$是可去间断点.$\lim_{x\to2^+}f(x)=0,\lim_{x\to2^-}f(x)=\infty$,$x=2$是第二类间断点.
    \end{solution}
    \begin{note}
        如何找间断点?主要是找可疑点
        \begin{itemize}
            \item \textcolor{red}{绝对值分段点}
            \item \textcolor{red}{这一点本身没有定义,但邻域内都有定义的点}\footnote{比如分母为0的点}.
        \end{itemize}
        $\ln(x)$本身不需要讨论$x$等于$0$,因为只有$0$点右邻域有定义,$0$点的左邻域内连定义都没有,更不用谈$0$点的左极限,所以此时$0$不可能是间断点.
        但出现$\ln |x|,\ln(x^2)$时,$0$点本身无定义,但$0$点左右邻域内都有定义,所以$0$可能是间断点.
    \end{note}
    \begin{problem}
        \getRating{3}函数$f(x)=\dfrac{\mid x\mid^{x}-1}{x(x+1)\ln\mid x\mid}$的可去间断点的个数为
    \end{problem}
    \begin{solution}
        $f(x)=\dfrac{\left|\:x\:\right|^{x}-1}{x(\:x+1)\ln\mid x\mid}\:$在 $x=-\:1,0,1$\:处无定义\:\\ 
        $\operatorname*{lim}_{x\to-1}f(x)=\operatorname*{lim}_{x\to-1}\dfrac{\mathrm{e}^{x\ln\left|x\right|}-1}{x(x+1)\ln\left|x\right|}=\operatorname*{lim}_{x\to-1}\dfrac{x\mathrm{ln}\left|x\right|}{x(x+1)\mathrm{ln}\left|x\right|}=\operatorname*{lim}_{x\to-1}\dfrac{1}{x+1}=\infty$, \\
        $\lim_{x\to0}f(x)=\lim_{x\to0}\dfrac{\left|x\right|^{x}-1}{x(x+1)\ln\left|x\right|}=\lim_{x\to0}\dfrac{\mathrm{e}^{x\ln\left|x\right|}-1}{x(x+1)\ln\left|x\right|}=\lim_{x\to0}\dfrac{x\ln\left|x\right|}{x(x+1)\ln\left|x\right|}=\lim_{x\to0}\dfrac{1}{x+1}=1\:, $\\
        $\operatorname*{lim}_{x\to1}f(x)\:=\operatorname*{lim}_{x\to1}{\dfrac{\left|\:x\:\right|^{x}-1}{x(x+1)\ln\left|\:x\:\right|}}=\operatorname*{lim}_{x\to1}{\dfrac{\mathrm{e}^{x\ln\left|\:x\:\right|}-1}{x(x+1)\ln\left|\:x\:\right|}}=\operatorname*{lim}_{x\to1}{\dfrac{x\ln\left|\:x\:\right|}{x(x+1)\ln\left|\:x\:\right|}}=\operatorname*{lim}_{x\to1}{\dfrac{1}{x+1}}={\dfrac{1}{2}}$
        综上,$x=0$与$x=1$为可去间断点
    \end{solution}
    \begin{defn}{跳跃间断点的定义}{}
        跳跃间断点\footnote{一点极限存在\neq f(x)在$x_0$连续}:若$\lim_{x\to x_0^-}f(x)$与$\lim_{x\to x_0^+}f(x)$都存在,但$\lim_{x\to x_0^+}f(x)\neq\lim_{x\to x_0^-}f(x)$,则这类间断点称为跳跃间断点
        \begin{center}
            \tikzset{every picture/.style={line width=0.5pt}} %set default line width to 0.5pt        
            \begin{tikzpicture}[x=0.5pt,y=0.5pt,yscale=-1,xscale=1]
            \draw  (152.86,148.43) -- (351.71,148.43)(251.95,67.87) -- (251.95,231.3) (344.71,143.43) -- (351.71,148.43) -- (344.71,153.43) (246.95,74.87) -- (251.95,67.87) -- (256.95,74.87) (271.95,143.43) -- (271.95,153.43)(291.95,143.43) -- (291.95,153.43)(311.95,143.43) -- (311.95,153.43)(331.95,143.43) -- (331.95,153.43)(231.95,143.43) -- (231.95,153.43)(211.95,143.43) -- (211.95,153.43)(191.95,143.43) -- (191.95,153.43)(171.95,143.43) -- (171.95,153.43)(246.95,128.43) -- (256.95,128.43)(246.95,108.43) -- (256.95,108.43)(246.95,88.43) -- (256.95,88.43)(246.95,168.43) -- (256.95,168.43)(246.95,188.43) -- (256.95,188.43)(246.95,208.43) -- (256.95,208.43) ;
            \draw   ;
            \draw    (250.09,108.43) -- (352.36,109.43) ;
            \draw   (250.09,108.43) .. controls (250.09,107.13) and (251.15,106.07) .. (252.45,106.07) .. controls (253.75,106.07) and (254.81,107.13) .. (254.81,108.43) .. controls (254.81,109.73) and (253.75,110.79) .. (252.45,110.79) .. controls (251.15,110.79) and (250.09,109.73) .. (250.09,108.43) -- cycle ;
            \draw  [fill={rgb, 255:red, 0; green, 0; blue, 0 }  ,fill opacity=1 ] (250.09,148.43) .. controls (250.09,147.4) and (250.92,146.57) .. (251.95,146.57) .. controls (252.97,146.57) and (253.81,147.4) .. (253.81,148.43) .. controls (253.81,149.45) and (252.97,150.29) .. (251.95,150.29) .. controls (250.92,150.29) and (250.09,149.45) .. (250.09,148.43) -- cycle ;
            \draw    (150.09,187.43) -- (252.36,188.43) ;
            \draw   (250,188.43) .. controls (250,187.13) and (251.06,186.07) .. (252.36,186.07) .. controls (253.66,186.07) and (254.71,187.13) .. (254.71,188.43) .. controls (254.71,189.73) and (253.66,190.79) .. (252.36,190.79) .. controls (251.06,190.79) and (250,189.73) .. (250,188.43) -- cycle ;
            \draw (246,43) node [anchor=north west][inner sep=0.75pt]   [align=left] {$\displaystyle y$};
            \draw (353,138) node [anchor=north west][inner sep=0.75pt]   [align=left] {$\displaystyle x$};
            \draw (233,149) node [anchor=north west][inner sep=0.75pt]   [align=left] {$\displaystyle O$};
            \draw (122,138) node [anchor=north west][inner sep=0.75pt]   [align=left] {$ $};
            \draw (181,236) node [anchor=north west][inner sep=0.75pt]   [align=left] {跳跃间断点函数图像};
            \end{tikzpicture}
        \end{center}
    \end{defn}
    \begin{defn}{无穷间断点的定义}{}
        无穷间断点:若$\lim_{x\to x_0}f(x)=\infty$,则这类间断点称为无穷间断点,如$y=\tan x$
              \begin{figure}[H]
                  \centering \includegraphics[width=
                      0.4 \linewidth]{1.3.6.eps} \caption{无穷间断点函数$\tan$图像}
              \end{figure}
    \end{defn}
    \begin{defn}{振荡间断点的定义}{}
        振荡间断点:若$\lim_{x\to x_0}f(x)$振荡不存在,则这类间断点称为振荡间断点
              \begin{figure}[H]
                  \centering \includegraphics[width=
                      0.4 \linewidth]{3.2.3.eps} \caption{振荡间断点函数$\sin \dfrac{1}{x}$图像}
              \end{figure}
    \end{defn}    
    \subsection{间断点的分类}
    通过求函数在该点的左右极限来判断
    \begin{itemize}
        \item 第一类间断点:$\lim _ { x \rightarrow x _ { 0 } ^{-}} f ( x )$​ 和$\lim _ { x \rightarrow x _ { 0 }^ {+}} f ( x )$​ 均存在
              \begin{itemize}
                  \item 可去\footnote{可去间断点上极限存在但是导数不存在}:$\lim _ { x \rightarrow x_0 ^ { - } } f ( x ) = \lim _ { x \rightarrow x_0^{+} } f  ( x ) \neq f(x_0)$
                  \item 跳跃:$\lim _ { x \rightarrow x_0^{-} } f ( x ) \not= \lim _ { x \rightarrow x_0^{+} } f ( x )$
              \end{itemize}
        \item 第二类间断点:除第一类以外的间断点$\implies \lim _ { x \rightarrow x _ { 0 } ^{-}} f ( x )$和$\lim _ { x \rightarrow x _ { 0 }^ {+}} f ( x )$​ 均至少一个不存在
    \end{itemize}
    \begin{problem}
        \getRating{2} 设函数 $f(x)=\lim_{n\to\infty}\dfrac{x^2+nx(1-x)\sin^2\pi x}{1+ n \sin^2 \pi x}$ ,则 $f(x)=$ 
    \end{problem}
    \begin{solution}
        分情况讨论,当$\sin^2 \pi x = 0$和$\sin^2 \pi x \neq 0$\\
        当$\sin^2 \pi x = 0$时,
        \begin{align*}
          \text{原式} & = \lim_{n\to \infty}\dfrac{x^2}{1} \\
          & =  x^2
        \end{align*}
        当$\sin^2 \pi x \neq 0$时,
        \begin{align*}
          \text{原式} & = \lim_{n\to \infty} \dfrac{\frac{x^2}{n}+x(1-x)\sin^2 \pi x}{\frac{1}{n}+\sin^2 \pi x} \\
          & = \lim_{n\to \infty} \dfrac{x(1-x)\sin^2 \pi x}{\sin^2 \pi x}\\
          & = x(1-x)
        \end{align*}
    综上函数$f(x)=\begin{cases}
        x^2 ,\sin^2 \pi x=0\\
        x(1-x),\sin^2 \pi x \neq 0
    \end{cases}$
    \end{solution}
    \begin{problem}
        \uline{求函数} $f(x)=\lim_{n\to\infty}\dfrac{x^{n+2}-x^{-n}}{x^n+x^{-n}}$ 的间断点并指出其类型.
    \end{problem}
    \begin{solution}
        当$n \to \infty$时,有$\lim_{x \to \infty}x^n=\begin{cases}
            \infty ,|x|>1\\ 
            0,|x|<1\\1,
            x=1\\
            (-1)^n,x=-1
        \end{cases}$,那么$\lim_{n\to\infty}\dfrac{x^{n+2}-x^{-n}}{x^n+x^{-n}}=
        \begin{cases} 
            -1,0<|x|<-1\\x^2,|x|>1\\0,|x|=1
        \end{cases}$
    综上,$x= \pm 1$为跳跃间断点,$x=0$为可去间断点.
    \end{solution}
    \begin{note}
        \textcolor{red}{对于$f(x)$是$x$的函数,表达式是以$n$的极限的形式给出的情况,方法为把$f(x)$分段解出来,$n$趋于无穷时,$x^n$要以$|x|=1$为界限进行分段.}\\同时应该结合$x^n$的解析式进行求解:$\lim_{x \to \infty}x^n=\begin{cases}
            \infty ,|x|>1\\ 
            0,|x|<1\\1,
            x=1\\
            (-1)^n,x=-1
        \end{cases}$
    \end{note}
    \begin{problem}
        \uline{设} $f(x)=\lim_{n\to\infty}\dfrac{2\mathrm{e}^{(n+1)x}+1}{\mathrm{e}^{nx}+x^{n}+1}$,则 $f(x):$  \\
        (A)仅有一个可去间断点.\quad(B) 仅有一个跳跃间断点.\quad(C) 有两个可去间断点.\quad(D)有两个跳跃间断点.
    \end{problem}
    \begin{solution}
        $\lim_{x\to \infty}e^{nx}=\begin{cases}
            0,x<0\\
            1,x=0\\
            +\infty,x>0
        \end{cases}$,
        $\lim_{x \to \infty}x^n=\begin{cases}
            \infty ,|x|>1\\ 
            0,|x|<1\\1,
            x=1\\
            (-1)^n,x=-1
        \end{cases}$
    \\综上可得:
    $
    \lim{n\to\infty}=f(x)=\lim_{n\to\infty}\dfrac{2\mathrm{e}^{(n+1)x}+1}{\mathrm{e}^{nx}+x^{n}+1}=\begin{cases}
        0,x<-1\\1,-1<x<0\\2e^x,x>0
    \end{cases}
    $
    \end{solution}
    %  ############################ 正文部分
        \ifx\allfiles\undefined
    \end{sloppypar}
\end{document}
\fi
