\ifx\allfiles\undefined
\documentclass[10pt, a4paper, oneside, UTF8]{ctexbook}  % +  这一句是新增加的
\input{../config/config}
\begin{document}
\begin{sloppypar}

    % \title{{\Huge{\textbf{考研高等数学笔记}}}}
\author{作者:LoafPhilosopher }
\date{\today}
\maketitle                   % 在单独的标题页上生成一个标题
\thispagestyle{empty}        % 前言页面不使用页码
% \begin{center}
% 	\Huge\textbf{前言}
% \end{center}
% 本笔记主要结合张宇老师和武忠祥老师的以及邂逅遗憾老师的内容进行编写
% \begin{flushright}
% 	\begin{tabular}{c}
% 		\today \newline 
% 	\end{tabular}
% \end{flushright}
\newpage                      % 新的一页
\pagestyle{plain}             % 设置页眉和页脚的排版方式(plain:页眉是空的,页脚只包含一个居中的页码)
\setcounter{page}{1}          % 重新定义页码从第一页开始
\pagenumbering{Roman}         % 使用大写的罗马数字作为页码

\begin{spacing}{1.5}
	\tableofcontents
\end{spacing}           % 生成目录
\newpage                      % 以下是正文
\pagestyle{plain}
\setcounter{page}{1}          % 使用阿拉伯数字作为页码
\pagenumbering{arabic}
% \setcounter{chapter}{-1}    % 设置 -1 可作为第零章绪论从第零章开始 
    \else
    \fi
    %  ############################ 正文部分
    \chapter{函数的微分}

    \section{微分的定义}
    \begin{defn}{微分的定义}{}
        设函数$y=f(x)$在某区间内有定义,$x_0$及$x_0+\Delta x$在这个区间内,如果函数的增量为
        $$
            \Delta y = f ( x _ { 0 } + \Delta x ) - f ( x _ { 0 } )
        $$
        可表示为
        $$
            \Delta y = A \Delta x + o ( \Delta x )
        $$
        其中$A$是不依赖于$\Delta x$的常数,那么称函数$f(x)$在点$x_0$是可微的,而$A\Delta x$叫做函数$y=f(x)$在点$x_0$相应于自变量增量$\Delta x$​的微分,记作$dy$,即:
        $$
            d y =  A\Delta x
        $$
        函数$f(x)$在任意点$x$的微分,称为函数的微分,记作$dy$或$df(x)$,即
        $$
            \textcolor{red}{d y = f ^ { \prime } ( x ) \Delta x}
        $$
    \end{defn}
    核心思想:局部用切线段近似代替曲线段
    \section{微分的几何意义}
    \begin{figure}[H]
        \centering \includegraphics[width=
            0.4 \linewidth]{2.3.1.pdf} \caption{函数微分的说明图像}
    \end{figure}
    函数在一点的微分.其中红线部分是微分量,而加上灰线部分后是实际的改变量
    \section{微分存在的意义}
    在当前仅知当前函数的$x$值和导数的情况下,预测未来函数的值,并推测出一个极限,来保证预测值与真实值相近
    \section{微分的计算公式}

    \section{近似计算}
    \begin{defn}{}{}
        如果$y=f(x)$在点$x_0$处的导数$f'(x_0)\neq 0$,且$|\delta x|$很小时,我们有
        $$
            \Delta y\approx\mathrm{d}y=f^{\prime}\left(x_{0}\right)\Delta x
        $$
        即函数在这个点的微分,那么函数在点$x$处的值可以近似为
        $$
            f(x)\approx f(x_{0})+f^{\prime}(x_{0})(x-x_{0})
        $$
        或
        $$
            f(x_0+\Delta x) \approx f(x_{0})+f^{\prime}(x_{0})(x-x_{0})
        $$
    \end{defn}
    %  ############################ 正文部分
    \ifx\allfiles\undefined
\end{sloppypar}
\end{document}
\fi
